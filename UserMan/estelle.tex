\subsection[Introduction]{Introduction}
 
Estelle is a {\em formal description technique} (FDT) based on
an extended state transition model (Estelle is a loose
acronym for {\em Extended State Transition Language}). It is
being developed as an International Standards Organisation
standard by ISO/TC 97/SC 21, the subcommittee for
Information Retrieval, Transfer and Management for the Open
Systems Interconnection (OSI).
 
Estelle can be used to describe distributed, concurrent
information processing systems. It is aimed in particular
at describing the service definitions and protocol
specifications of OSI layers. In the OSI model, each layer
is separated from the layers above and below, and
communicates with these via service requests. Apart from
this communication, the various layers can be regarded as
separate asynchronous systems. Furthermore, the peer layers
on separate systems behave relatively asynchronously with
respect to one another. A further characteristic of such
systems is nondeterminism. Through a hierarchical structure
of processes, Estelle provides facilities for both
asynchronous parallel execution and nondeterministic
sequential execution.
 
Essentially, Estelle is a subset of the programming
language Pascal (ISO 7185) with extensions for concurrency
based on an extended finite-state model. An Estelle program
(called a {\bf specification}) specifies a number of processes
which may execute either sequentially or concurrently, or a
combination of both. Each process is an extended finite-
state automaton, consisting of a number of {\em states}, a set of
possible {\em state transitions}, the {\em conditions} which must be
satisfied for the transitions to be {\em enabled}, and the 
{\em actions} associated with each transition. The automata communicate
with one another using {\em message passing} on FIFO queues. Each
automaton typically represents a layer or part thereof of a
protocol executing on some specific machine. A 3-layer
protocol executing on two separate machines would thus
typically be represented in Estelle using 3*2=6 automata.
The entire system is dynamic and nondeterministic; new
processes or automata may be freely created and destroyed
during the execution of a specification.
 
 This informal introduction to the syntax
and semantics of Estelle assumes the reader is familiar
with the Pascal language; only the changes and extensions to
Pascal are considered.

\subsection[Conventions and Outline]{Conventions and Outline}
\label{conventions}
 
 Several conventions have been used throughout this
document. Within syntactic grammar rules, bold type has
been used to represent terminals, italic type for
nonterminals, and normal type for the grammar metalanguage.
The metalanguage consists of the symbols:

\begin{tabbing}
mmmmmm\=mmmmmmm\=\+\kill
=\>Separates right and left hand sides of productions\\
$|$\>Separates alternatives\\
$[]$\>Indicates optional elements\\
\{\}\>Indicates zero or more elements\\
+\{\}\>Indicates one or more elements\\
()\>Used for grouping purposes\\
$<>$\>Used to identify nonterminals\\
\dots\>Used to abbreviate a scalar range\\
\end{tabbing}

Unconventionally, productions are not terminated with a
period as this may cause confusion with punctuation symbols
that are intended as language terminals.
 
Within the productions, upper case nonterminals represent
Pascal constructs. In general, these productions are not
explicitly given, but should be fairly obvious to the reader
who is familiar with Pascal. Lower case nonterminals are
used exclusively for Estelle constructs.
 
Within the text, important terms have been indicated in
italic, while grammar terminals are once again indicated in
bold.
 
To clarify the purpose of identifiers, they are often
prefixed with an abbreviation to indicate their use. For
example, {\em role-IDENT} is an identifier whose purpose it is to
identify a role. These prefixes are included for semantic
clarification only; all identifiers ultimately have the same
production rule.

The various parts of a programming language are usually
strongly interconnected; it is difficult to discuss any one
aspect of the language in isolation from others. The
breakdown used here is necessarily artificial.
In an attempt to reduce forward references different parts
of the language will be introduced as they are used;
however, the detailed discussion of any particular part will
be confined to the relevant section.
 
The first section is a brief overview of the additions to
ISO Pascal. None of the additions are discussed in detail;
they are dealt with individually in the remaining sections.
Section \ref{restrict} considers the restrictions to ISO Pascal that
have been introduced.
 
Section \ref{terminology} introduces some important terminology, while
Section \ref{automata} is a discussion of how finite-state automata may
be realised using Estelle. It concentrates primarily on the
external behaviour of the automata, regarding each automaton
as a process in a multiprocess system. Section \ref{states}
discusses the internals of the automata, namely, the
specification of states and state transitions, and how these
are interpreted during execution. Section \ref{IPC} concentrates
on the communication between automata, that is, the message
passing and shared variable facilities of Estelle.
 
As each new aspect of the language is covered, the
syntactic production rules for that aspect are given, and
the semantics discussed. A collected syntax is included in
the appendix. Section \ref{conventions} describes the metalanguage used to
specify the syntactic productions, as well as other
conventions used.

\subsection[Additions to ISO Pascal]{Additions to ISO Pascal}

The main additions to standard Pascal are the provision
for concurrent systems and the provision for specifying
finite-state automata. Other additions include:

\begin{itemize}
\item identifier names may have underscores in any position;
this is the only difference from standard Pascal
identifiers. The general production rule for identifiers
is thus:

\begin{tabbing}
mmm\=mmmmmmm\=mm\=\+\kill
$<$IDENT$>$\>=\>$<$letter$>$ \{ $<$letter$>|<$digit$>$ \}\\
$<$letter$>$\>=\> a $|$ b $|$ c $|$\dots$|$ z $|$ A $|$ B $|$ C $|$\dots$|$ Z $|$ \_\\
$<$digit$>$\>=\>0 $|$ 1 $|$ 2 $|$\dots$|$ 9
\end{tabbing}

\item partial specifications are possible, allowing early
syntactic checking, separate compilation or the 
specification of protocols without specifying irrelevant details.
This is achieved by allowing constants to be declared with the 
value {\bf any} and variables to be declared with the type 
\dots. In addition to functions and procedures being able to be
forward-declared by qualifying them with the keyword {\bf forward},
the {\bf primitive} and {\bf external} keywords can be similarly
used to declare functions or procedures that are not defined 
within the particular file containing these definitions. An {\bf
external} routine must be declared within another file which is part
of the specification; a {\bf primitive} routine is one which is
provided by the language implementor (typically a library
function/procedure). 

This facility allows incomplete specifications to
be compiled and thus checked for syntactic errors,
access to external libraries of functions, as well as
the ability to omit irrelevant parts of the protocol
specification (for example, the type of a protocol
data unit is not needed to define the protocol
itself).

\item the {\bf all} repetitive statement has been added to the
existing {\bf repeat}, {\bf while} and {\bf for} statements.

\item the structured statement {\bf forone} has been added

\item the expression factor {\bf exist} has been added

\item the simple statements {\bf attach}, {\bf detach}, 
{\bf connect}, {\bf disconnect}, {\bf output}, {\bf init},
{\bf release} and {\bf terminate} have been added.
\end{itemize}

The {\bf init}, {\bf release}, {\bf terminate}, {\bf all} and
{\bf forone} statements and {\bf exist} expression are discussed 
in Section \ref{automata}; the {\bf attach}, {\bf detach},
{\bf connect}, {\bf disconnect}, and {\bf output} statements
are discussed in Section \ref{IPC}.

\subsection[Restrictions to ISO Pascal]{Restrictions to ISO
Pascal}
\label{restrict}

The fact that Estelle supports concurrency has
repercussions within the Pascal subset, primarily to allow
functions and procedures to be reentrant. Several
restrictions to procedures and functions have thus been
introduced. They may not reference non-Pascal objects,
although the Estelle statements {\bf all}, {\bf forone} and {\bf
exist} may
still be used on the usual Pascal ordinal types. Functions
must all be pure; that is, side-effects are prohibited.
This is enforced through several restrictions: parameters
may be passed by value only and may not contain any
pointers, references to global variables are prohibited and
so are calls to any other non-pure routines. Procedures
need not be pure; if they are, they may be declared as {\bf pure
procedure}, and may then be called from functions and other
pure procedures.
 
All file operations have been removed, including the file
type and predefined {\bf text} type. Thus the execution of an
Estelle specification is only meaningful in a controlled
environment, or if the implementor adds his or her own
{\bf primitive} library routines for input/output.
 
Labels and {\bf goto}s are also restricted - they may only
occur within procedures and functions, and labels may only
be associated with the end of the containing procedure or
function; in other words, a {\bf goto} acts similar to a {\bf
return} statement in a language such as 'C'.

\subsection[Terminology]{Terminology}
\label{terminology}

Several terms are used within Estelle to describe
processes and interprocess communication. Although these
terms are discussed within the relevant sections, it is
appropriate to introduce some here.
 
A {\em channel} is a bidirectional FIFO queue used for
interprocess communication. Each channel is identified by a
{\em channel identifier}. The traffic over a channel (which
consists of messages called {\em interactions}) can be different
for each of the two directions; each direction is associated
with a {\em channel role} which specifies what interactions may be
sent in that direction. The endpoints of a channel are
called {\em interaction points}.
 
An executing protocol {\em specification} in Estelle consists
of one or more {\em modules} in a hierachical structure. These
modules may have different {\em classes}: {\em activities} allow
nondeterministic sequential execution (ie, if the scheduler
must choose between a number of enabled activities, it will
choose only one, and the choice is nondeterministic), while
{\em processes} allow asynchronous parallel execution (ie, the
scheduler will execute all of the enabled processes). This
description is an oversimplification of what actually
occurs, but is sufficient for now.
 
Each module comprises two parts: the {\em module header}
specifies the external visibility of the module, while the
{\em module body} specifies its internal behaviour.
 
The conditions associated with transitions are called
{\em clauses}, and may be of several different types.

\subsection[Automata]{Automata}
\label{automata}

In this section, we will consider automata as processes
in a multiprocessing system. We are thus primarily
concerned with the creation and destruction of automata, not
with their internal behaviour or intercommunication, which
are discussed in sections \ref{states} and \ref{IPC} respectively.
 
At the top level, an Estelle program is a
$<${\em specification}$>$, identified by a name $<${\em spec-IDENT}$>$ and a
class $<${\em system-class}$>$, with an internal structure described
by a $<${\em body-defn}$>$. Queues for interprocess communication may
be {\em individual} or {\em shared}; usually such {\em queue
disciplines} will
be explicitly declared within queue declarations, although a
global default can be specified with the optional
$<${\em defaults}$>$. This default must be provided if there are any
queues declared without explicit queue disciplines.
 
Timescales are semantically meaningless and are simply
used to express the intentions of the implementor. Time in
the Estelle model need only be treated consistently; the
actual units are irrelevant (except in real
implementations), and serve only to express the intentions
of the implementor.

\begin{tabbing}
mm\=mmmmmmmm\=mm\=\+\kill
$<${\em specification}$>$\>=
\>{\bf specification} $<${\em spec-IDENT}$>$ [$<${\em system-class}$>$]{\bf
;}\\
\>\>[ $<${\em defaults}$>$ ]\\
\>\>[ $<${\em time-options}$>$ ]\\
\>\>$<${\em body-defn}$>$\\
\>\>{\bf end .}\\
\mbox{}\\
$<${\em system-class}$>$\>=\>{\bf systemprocess} $|$ {\bf systemactivity}\\
\mbox{}\\
$<${\em defaults}$>$\>=\>{\bf default} $<${\em q-discipline}$>$ {\bf ;}\\
\mbox{}\\
$<${\em q-discipline}$>$\>=\>{\bf common queue} $|$ {\bf individual queue}\\
\mbox{}\\
$<${\em time-options}$>$\>=\>{\bf timescale} $<${\em ts-IDENT}$>$ {\bf ;}\\
\mbox{}\\
$<${\em ts-IDENT}$>$\>=\>{\bf hours} $|$ {\bf minutes} $|$ {\bf seconds} $|$\\
\>\>{\bf milliseconds} $|$ {\bf microseconds}\\
\mbox{}\\
$<${\em body-defn}$>$\>=\>$<${\em decl-part}$>$\\
\>\>$<${\em init-part}$>$\\
\>\>$<${\em trans-decls}$>$\\
\end{tabbing}

The $<${\em body-defn}$>$ of a specification consists of three
parts: the {\em declarations}, an {\em initialisation part}, and a
{\em transition declaration part}. The initialisation and
transition declaration parts both specify groups of
transitions (see section \ref{states} for details). Initially it is
assumed that the specification exists in an initial state
that results from executing a transition from its $<${\em init-
part}$>$ if this is non-empty; otherwise it is in a {\em pre-initial
state}.

\begin{tabbing}
mm\=mmmmmmmm\=mm\=\+\kill
$<${\em decl-part}$>$\>=\>{ $<${\em decls}$>$ }\\
\mbox{}\\
$<${\em decls}$>$\>=\>$<${\em CONSTANT-DEFN-PART}$>$\\
\>\>$|<${\em TYPE-DEFN-PART}$>$\\
\>\>$|<${\em channel-defn}$>$\\
\>\>$|<${\em module-header-defn}$>$\\
\>\>$|<${\em module-body-defn}$>$\\
\>\>$|<${\em ip-decls}$>$\\
\>\>$|<${\em mod-var-decls}$>$\\
\>\>$|<${\em VARIABLE-DECL-PART}$>$\\
\>\>$|<${\em state-defns}$>$\\
\>\>$|<${\em state-set-defns}$>$\\
\>\>$|<${\em PROCEDURE-AND-FUNCTION-DECL-PART}$>$\\
\mbox{}\\
$<${\em init-part}$>$\>=\>\{ {\bf initialise} $<${\em trans-group}$>$ \}\\
\mbox{}\\
$<${\em trans-decls}$>$\>=\>\{ $<${\em trans-decl}$>$ \}\\
\mbox{}\\
$<${\em trans-decl}$>$\>=\>{\bf trans} $<${\em trans-group}$>$\\
\mbox{}\\
$<${\em trans-group}$>$\>=\>+\{ $<${\em clause-group}$><${\em trans-block}$>$
{\bf ;} \}\\
\end{tabbing}

The declarations may be in any order, and, with the
exception of the $<${\em state-defn-part}$>$, any declaration or
definition part may occur more than once. Apart from the
usual Pascal objects, the declarations may contain {\em channels}
(bidirectional FIFO message queues) and {\em interaction points}
(queue endpoints), module {\em headers}, {\em bodies} and {\em
variables}, and {\em states} and {\em state sets}.

An executing Estelle specification consists of
a tree of {\em instances} of generic {\em modules}. The
process structure may be dynamic; processes may at any time
instantiate others with {\bf init} statements and kill children
processes with the {\bf release} or {\bf terminate} statements. These
are discussed later in this section.

An Estelle module is described by its external visibility
(which is defined by the $<${\em modheader-defn}$>$) and internal 
behaviour (which is defined by the $<${\em modbody-defn}$>$).
These two components are separate, and thus modules with the
same external visibility may exhibit different internal
behaviour. Modules are bound to headers at compile time and bodies at
instantiation time.

\begin{tabbing}
mm\=mmmmmmmmmm\=mm\=\+\kill
$<${\em modheader-defn}$>$\>=
\>{\bf module} $<${\em modhdr-IDENT}$>$ [ $<${\em class}$>$ ]\\
\>\>[ {\bf (} $<${\em param-list}$>$ {\bf )} ] {\bf ;}\\
\>\>[ {\bf ip} +\{ $<${\em ip-decl}$>$ {\bf ;} \} ]\\
\>\>[ {\bf export} +\{ $<${\em VARIABLE-DECL}$>$ {\bf ;} \} ]\\
\>\>{\bf end ;}\\
\mbox{}\\
$<${\em class}$>$\>=\>{\bf systemprocess} $|$ {\bf systemactivity}\\
\>\>$|$ {\bf process} $|$ {\bf activity}\\
\mbox{}\\
$<${\em param-list}$>$\>=\>$<${\em VALUE-PARAM-SPECIFICATION}$>$\\
\>\>\{ {\bf ;} $<${\em VALUE-PARAM-SPECIFICATION}$>$ \}
\end{tabbing}

The only external access to a module instance is via its
interaction points and {\em exported variables} which are shared
with the instance's parent only. These exported variables
are declared in the $<${\em modheader-defn}$>$. The module header
also defines parameters which are passed (by value only) to
an instance of the module when the instance is created. The
interaction points declared in the module header are
{\em external interaction points}; in other words, they may be
accessed by the module's parent.

\begin{tabbing}
mm\=mmmmmmmm\=mm\=\+\kill
$<${\em modbody-defn}$>$=\>\>{\bf body} $<${\em modbod-IDENT}$>$ {\bf for} $<${\em
modhdr-IDENT}$>$ {\bf ;}\\
\>\>( $<${\em body-defn}$>$ {\bf end ;}\\
\>\>$|$ {\bf external ;} )
\end{tabbing}

Notice that two identifiers are used to identify the
module body; the first is associated with the module body
itself, and the second specifies which module header is
being referenced. If the body definition contains the
keyword {\bf external}, then the actual body definition is in some
other specification.

The relative behaviour of module instances in the
hierarchy is determined by the nesting of their definitions
as well as the class qualifiers {\bf systemprocess},
{\bf systemactivity}, {\bf process} and {\bf activity}. Processes allow for
synchronous parallel execution, while activities allow
nondeterministic sequential execution. A process may be
created and released only by its parent. A module with a
class qualifier is called {\em attributed}; the only non-
attributed modules are inactive modules, which as their name
suggests consist only of initialisation parts. Inactive
modules may only parent system modules, which may in turn
parent only process or activity modules. (System)process
modules may contain either process or activity children,
while (system)activity modules may contain only other
activity modules.

A consequence of the above is that any initial state of a
specification defines a fixed number of system instances and
interaction point links above them; this structure once
initialised cannot change since the parent modules of the
systems are all inactive and hence cannot create any new
processes after initialisation. Furthermore, as their
parents are inactive and hence have no execution priority
over the systems, the systems behave fully asynchronously
with respect to one another.

Instances of modules are identified by {\em module variables}.
References to exported variables and interaction points must
be qualified by the name of the appropriate module variable.
Arrays of module variables are allowed.

\begin{tabbing}
mm\=mmmmmmmm\=mm\=mmmm\=\+\kill
$<${\em mod-var-decls}$>$=\>\>{\bf modvar} +\{ $<${\em mod-var-decl}$>$ {\bf
;} \}\\
\mbox{}\\
$<${\em mod-var-decl}$>$\>=\>$<${\em IDENT-LIST}$>$ {\bf :} $<${\em
modhdr-IDENT}$>$\\
\>\>$|<${\em IDENT-LIST}$>$ {\bf : array [} $<${\em index-type-list}$>$ {\bf
]}\\
\>\>\>{\bf of} $<${\em modhdr-IDENT}$>$\\
\end{tabbing}

The initial value of a module variable is undefined; an
{\bf init} statement will initialise the variable, and a {\bf
release} or {\bf terminate} statement causes it to once 
again become undefined.

\begin{tabbing}
mm\=mmmmmmmmmmmm\=mm\=mmm\=\+\kill
$<${\em init-stmnt}$>$\>=\>{\bf init} $<${\em module-var}$>$ {\bf with} $<${\em
modbod-ident}$>$\\
\>\>\>[ {\bf (} $<${\em EXPR-LIST}$>$ {\bf )} ]\\
\mbox{}\\
$<${\em release-stmnt}$>$\>=\>{\bf release} $<${\em module-var}$>$\\
\mbox{}\\
$<${\em terminate-stmnt}$>$\>=\>{\bf terminate} $<${\em module-var}$>$
\end{tabbing}

The execution of the {\bf init} statement causes a new instance
of the specified module body (and thus header) to be
created, its parameters to be initialised with the
expression values, and an initialisation transition from the
module body (if one exists and is enabled) to be executed.
After execution of the {\bf init} statement, the module variable
identifies the module instance.
 
The {\bf release} statement allows a module to release children
module instances. All external interaction points of the
module instance are detached or disconnected, and the module
instance and {\em all its descendants} are released and are no
longer available. The {\bf terminate} statement is a similar
statement, except that it uses a form of detach known as
{\em simple detach}. This will be described later.
 
The {\bf all} statement is a repetitive statement which allows
iteration over a normal ordinal type or over a set of module
instances.

\begin{tabbing}
mm\=mmmmmmmmmmmm\=mm\=mmm\=\+\kill
$<${\em all-stmnt}$>$\>=\>{\bf all} ($<${\em domain-list}$>|<${\em
module-domain}$>$)\\
\>\>\>{\bf do} $<${\em STATEMENT}$>$\\
\mbox{}\\
$<${\em domain-list}$>$\>=\>$<${\em IDENT-LIST}$>$ {\bf :} $<${\em
ORDINAL-TYPE}$>$\\
\>\>\>\{ {\bf ;} $<${\em IDENT-LIST}$>$ {\bf :} $<${\em ORDINAL-TYPE}$>$
\}\\
\mbox{}\\
$<${\em module-domain}$>$\>=\>$<${\em IDENT}$>$ {\bf :} $<${\em modhdr-IDENT}$>$
\end{tabbing}

The result of an {\bf all} statement is the execution of the
$<${\em STATEMENT}$>$ for either:

\begin{itemize}
\item all vectors of values of ordinal type(s) given in the
 $<${\em domain-list}$>$; or

\item all children module instances whose header definitions are
 identified by the $<${\em modhdr-IDENT}$>$ in the $<${\em module-
 domain}$>$.
\end{itemize}

The order of execution is arbitrary. If the domain is
empty, the statement is not executed. The bounds of a
domain are evaluated once and are not affected by execution
of the statement.
 
A similar statement is the {\bf forone} statement:

\begin{tabbing}
mm\=mmmmmmmmm\=mm\=mmmm\=\+\kill
$<${\em forone-stmnt}$>$\>=\>
{\bf forone} ($<${\em domain-list}$>|<${\em module-domain}$>$)\\
\>\>\>{\bf suchthat} $<${\em BOOLEAN-EXPR}$>$\\
\>\>\>{\bf do} $<${\em STATEMENT}$>$\\
\>\>\>[ {\bf otherwise} $<${\em STATEMENT}$>$ ]
\end{tabbing}

This is much like the {\bf all} statement, except that only one
of the possible (vector of) values is used, and this must
satisfy the Boolean expression.
 
To assist in these statements, the {\bf exist} Boolean
expression may be used:

\begin{tabbing}
mm\=mmmmmmmm\=mm\=mmmm\=\+\kill
$<${\em exist-one}$>$\>=\>{\bf exist} ($<${\em domain-list}$>|<${\em
module-domain}$>$)\\
\>\>\>{\bf suchthat} $<${\em FACTOR}$>$\\
\end{tabbing}

The semantics are obvious in the light of the {\bf forone}
statement: if an element of the domain exists that satisfies
the $<${\em FACTOR}$>$, then the expression returns TRUE, otherwise it
returns FALSE.

\subsection[States and State Transitions]{States and State
Transitions}
\label{states}

We now consider the details of how Estelle allows state
transitions to be specified.
 
Each instance of a module is a nondeterministic state
transition system, characterised by a set of states, subset
of initial states, and next-state relation. The initial
states are defined by the module $<${\em init-part}$>$, which
describes the initial state, variable initialisation, and
the initial hierarchy and interconnection structure of
descendant module instances, if any. The one-to-many next-state
relation is specified by transitions, each composed of
a $<${\em clause-group}$>$ determining the pre- and post- states and
any conditions which must be fulfilled for the transition to
be {\em enabled}, and a $<${\em trans-block}$>$ defining the actions to be
executed upon firing the transition. If the transition
block is empty, the module is {\em inactive} and serves only as a
parent to other modules.

\begin{tabbing}
mm\=mmmmmmmm\=mm\=mmmm\=\+\kill
$<${\em init-part}$>$\>=\>\{ {\bf initialise} $<${\em trans-group}$>$ \}\\
\mbox{}\\
$<${\em trans-group}$>$\>=\>+\{ $<${\em clause-group}$><${\em trans-block}$>$
{\bf ;} \}\\
\end{tabbing}

All possible values of the control state of the EFSM must
be enumerated in a state definition part:

\begin{tabbing}
mm\=mmmmmmmm\=mm\=mmmm\=\+\kill
$<${\em state-defns}$>$\>=\>{\bf state} $<${\em IDENT-LIST}$>$ {\bf ;}
\end{tabbing}

States may be grouped together into state sets; this is
simply a compact notation allowing a single {\em state set
identifier} to reference a list of elements.

\begin{tabbing}
mm\=mmmmmmmmmmmmm\=mm\=mmmm\=\+\kill
$<${\em state-set-defns}$>$\>=\>{\bf stateset} +\{ $<${\em state-set-defn}$>$ {\bf ;}
\}\\
\mbox{}\\
$<${\em state-set-defn}$>$\>=\>
$<${\em state-set-IDENT}$>$ {\bf =} $<${\em state-set-constant}$>$\\
\mbox{}\\
$<${\em state-set-constant}$>$\>=\>[ $<${\em state-IDENT}$>$ \{ {\bf ,}
 $<${\em state-IDENT}$>$ \} ]\\
\end{tabbing}

Transitions consist of $<${\em clause-group}$>$s and 
$<${\em trans-block}$>$s:

\begin{tabbing}
mm\=mmmmmmmmmmmmm\=mm\=mmmm\=\+\kill
$<${\em trans-decl-part}$>$\>=\>\{ $<${\em trans-decl}$>$ \}\\
\mbox{}\\
$<${\em trans-decl}$>$\>=\>{\bf trans} $<${\em trans-group}$>$\\
\mbox{}\\
$<${\em trans-group}$>$\>=\>+\{ $<${\em clause-group}$><${\em trans-block}$>$ {\bf ;}
\}\\
\mbox{}\\
$<${\em clause-group}$>$\>=\>[ $<${\em provided-clause}$>$ ]\\
\>\>[ $<${\em from-clause}$>$ ]\\
\>\>[ $<${\em to-clause}$>$ ]\\
\>\>[ $<${\em any-clause}$>$ ]\\
\>\>[ $<${\em delay-clause}$>$ ]\\
\>\>[ $<${\em when-clause}$>$ ]\\
\>\>[ $<${\em priority-clause}$>$ ]\\
\mbox{}\\
$<${\em trans-block}$>$\>=\>$<${\em CONSTANT-DEFN-PART}$>$\\
\>\>$<${\em TYPE-DEFN-PART}$>$\\
\>\>$<${\em VAR-DECL-PART}$>$\\
\>\>$<${\em PROC-AND-FUNC-DECL-PART}$>$\\
\>\>[ $<${\em trans-name}$>$ ] $<${\em STATEMENT-PART}$>$\\
\mbox{}\\
$<${\em trans-name}$>$\>=\>{\bf name} $<${\em IDENT}$>$ {\bf :}\\
\end{tabbing}

The ordering of the clauses is not important.
Transitions with {\bf when} clauses are called {\em input
transitions}
as they are enabled only when some input condition occurs
(i.e. some message is received); those without are {\em spontaneous
transitions}. {\bf when} and {\bf delay} clauses are mutually exclusive.
The optional $<${\em trans-name}$>$ is for documentation purposes
only, and has no semantic effect.

Notice that the $<${\em trans-decl-part}$>$ consists of zero or more
$<${\em trans-decl}$>$s, each of which consists of a number of
transitions preceded by the keyword {\bf trans}. This grouping is
important, and affects two aspects: the nesting of clauses, and the
interpretation of {\bf provided otherwise} clauses. We shall discuss
the latter below. The nesting of clauses allows us to abbreviate
transitions which have clauses in common. Clauses are held on a
stack, which is affected in the following ways:

\begin{itemize}
\item upon entering a new $<${\em trans-decl}$>$, the stack is cleared.
\item upon entering a new $<${\em clause-group}$>$, the stack is popped
up to and including the clause type corresponding to the first clause
in the $<${\em clause-group}$>$. For example, if the new $<${\em
clause-group$>$} begins with a {\bf provided} clause, the stack will be
popped up to and including the {\bf provided} clause on the stack, if
any. If no such clause exists, the entire stack is popped.
\item subsequent to this, all clauses in the new $<${\em clause-group}$>$
are pushed on to the stack, one by one.
\item Finally, when the $<${\em trans-block}$>$ is entered, all clauses
currently on the stack are applicable to that block.
\item Once clauses are being pushed on to the stack, no duplicates
are allowed. For example, if a {\bf provided} clause is already
stacked, the occurence of a {\bf provided} clause is illegal.
\end{itemize}

We will now examine each type of clause in more detail.

\begin{tabbing}
mm\=mmmmmmmmmmmm\=mm\=mmmm\=\+\kill
$<${\em provided-clause}$>$\>=\>{\bf provided} ($<${\em BOOLEAN-EXPR}$>|$ {\bf otherwise})
\end{tabbing}

The {\bf provided} clause specifies a Boolean condition which
must be met for the transition to be enabled. If it is
omitted, it is assumed TRUE. {\bf otherwise} is only permitted as
the last clause of a transition group; it is TRUE is all other
{\bf provided} clauses in the transition group are FALSE.

\begin{tabbing}
mm\=mmmmmmmmmmmm\=mm\=mmmm\=\+\kill
$<${\em from-clause}$>$\>=\>{\bf from} $<${\em from-list}$>$\\
\mbox{}\\
$<${\em from-list}$>$\>=\>$<${\em from-element}$>$ \{ {\bf ,} $<${\em from-element}$>$
\}\\
\mbox{}\\
$<${\em from-element}$>$\>=\>$<${\em state-IDENT}$>|<${\em
state-set-IDENT}$>$\\
\mbox{}\\
$<${\em to-clause}$>$\>=\>{\bf to} $<${\em to-element}$>$\\
\mbox{}\\
$<${\em to-element}$>$\>=\>{\bf same} $|<${\em state-IDENT}$>$
\end{tabbing}

The $<${\em from-list}$>$ in a $<${\em from-clause}$>$ specifies those states
from which a transition may be validly executed; if it is
omitted, it is assumed satisfied. The $<${\em to-element}$>$
specifies the next state after the execution of the
transition; if it is omitted or same then the state does not
change.

\begin{tabbing}
mm\=mmmmmmmmmmmm\=mm\=mmmm\=\+\kill
$<${\em when-clause}$>$\>=\>{\bf when} $<${\em when-ip-ref}$>$ {\bf .} $<${\em
interaction-IDENT}$>$\\
\>\>\>[ $<${\em interaction-arg-list}$>$ ]\\
\mbox{}\\
$<${\em when-ip-ref}$>$\>=\>$<${\em ip-IDENT}$>$ [ {\bf [} $<${\em ip-index}$>$
\{ {\bf ,} $<${\em ip-index}$>$ \} {\bf ]} ]\\
\mbox{}\\
$<${\em ip-index}$>$\>=\>$<${\em CONSTANT}$>|<${\em VAR-IDENT}$>$\\
\mbox{}\\
$<${\em interaction-arg-list}$>$\>=\>( $<${\em interaction-arg-IDENT}$>$\\
\>\>\>\{ {\bf ,} $<${\em interaction-arg-IDENT}$>$ \} )\\
\end{tabbing}

A {\bf when} clause is satisfied if the interaction specified
by the $<${\em interaction-IDENT}$>$ is at the head of the queue
associated with the interaction point indicated by the
$<${\em when-ip-ref}$>$. The interaction at the head of the queue is
only dequeued as part of the execution of the transition.
The $<${\em interaction-arg-list}$>$ is optional; if it is included,
it must be complete and in the same order as in the
corresponding $<${\em interaction-defn}$>$. Section \ref{IPC} discusses
interactions in more detail.

\begin{tabbing}
mm\=mmmmmmmmmmmm\=mm\=mmmm\=\+\kill
$<${\em delay-clause}$>$\>=\>{\bf delay (} ($<${\em EXPR}$>$ {\bf ,} $<${\em
EXPR}$>$\\
\>\>\>$|<${\em EXPR}$>$ {\bf , *}\\
\>\>\>$|<${\em EXPR}$>$)\\
\>\>\>{\bf )}
\end{tabbing}

Some transitions may contain {\bf delay} clauses specifying a
minimum and maximum delay time; provided these are
interpreted consistently time in the Estelle model is
considered implementation dependent. A {\bf delay} clause
 
\begin{tabbing}
mm\=mmmmmmmm\=mm\=mmmm\=\+\kill
\>{\bf delay (E1,E2)}\\
\end{tabbing}
 
specifies that the transition should not be selected for
execution before E1 consecutive time-units of it being
enabled have elapsed. Once E1 has elapsed, the transition
may or may not be selected for execution (due to
nondeterminism), even if it is the only transition enabled.
However, once E2 units have elapsed, if the transition is
the only one enabled it must be selected. ``{\bf delay (E)}'' is
the same as ``{\bf delay (E,E)}''. An asterisk for the second
expression indicates that there is no upper bound, so it is
possible the transition may never be selected for execution.

\begin{tabbing}
mm\=mmmmmmmmmmm\=mm\=mmmm\=\+\kill
$<${\em priority-clause}$>$\>=\>{\bf priority} $<${\em priority-constant}$>$\\
\mbox{}\\
$<${\em priority-constant}$>$\>=\>$<${\em UNSIGNED-INTEGER}$>|<${\em
CONSTANT-IDENT}$>$\\
\end{tabbing}

{\bf priority} clauses are used to attach priorities to
transitions. Note that priorities are only relevant amongst
peer processes, as parent transitions always have priority
over children transitions. If the {\bf priority} clause is
omitted, the lowest priority is assumed. Zero is the
highest priority, after which they decrease systematically.

\begin{tabbing}
mm\=mmmmmmmm\=mm\=mmmm\=\+\kill
$<${\em any-clause}$>$\>=\>{\bf any} $<${\em domain-list}$>$ {\bf do}\\
\mbox{}\\
$<${\em domain-list}$>$\>=\>$<${\em IDENT-LIST}$>$ {\bf :} $<${\em
ORDINAL-TYPE}$>$\\
\>\>\>\{ {\bf ;} $<${\em IDENT-LIST}$>$ {\bf :} $<${\em ORDINAL-TYPE}$>$
\}\\
\end{tabbing}

{\bf any} clauses provide a shorthand way of specifying a
number of clauses. A transition with an any clause is
equivalent to as many transitions as there are distinct
values in the $<${\em ORDINAL-TYPE}$>$, one for each possible value.
The identifiers in an {\bf any} clause will typically be array
indexes used to refer to arrays of interaction points in
{\bf provided} clauses.
 
The EFSM represented by a module body has all its initial
states defined in the initialisation part:

\begin{tabbing}
mm\=mmmmmmmm\=mm\=mmmm\=\+\kill
$<${\em init-part}$>$\>=\>\{ {\bf initialize} $<${\em trans-group}$>$ \}\\
\end{tabbing}

The initialisation transition group is executed only once
when a module instance is created. Only one transition
block is executed, even if several are enabled (the choice
being nondeterministic). Only {\bf to} and {\bf provided} clauses are
allowed in the initialisation part.

As was mentioned in the previous section, systems allow
for synchronous parallel execution, while activities allow
nondeterministic sequential execution. The fact that we
have both concurrency and nondeterminism means that at any
stage, several transitions may be offered for execution. A
transition offered by a parent always has priority over one
offered by a child. If a parent offers no transition but
its children do, we first determine, for each child, the
transition(s) selected for execution. If the parent is a
(system)process then all of these transitions are selected
for execution; otherwise one child is selected (the choice
is nondeterministic) and its transition(s) used. This may
be made more precise as follows:

Consider the process hierarchy as a tree. The
transitions selected for execution ({\em $AS(gid_{P})$}) may be
synthesised bottom up using the following definition for any
particular non-leaf node {\em $P$} with children {\em $P_{i}$}:
 
\begin{itemize}
\item If {\em $P$} is active and offers a non-null transition, then this
 transition is selected (Parent priority).
\item Otherwise, if {\em $P$} is an (system)activity, then {\em
$AS(gid_{P})$} is one of the {\em $AS(gid_{P_{i}})$} (the choice is
nondeterministic).
\item Otherwise, if {\em $P$} is a (system)process, then {\em
$AS(gid_{P})$} is the union of all the {\em $AS(gid_{P_{i}})$} (parallel execution).
\end{itemize}
 
The execution of a specification thus proceeds as
follows: for each system, a set of transitions are selected
for execution. Once selected, a transition must eventually
execute. Once all of the selected transitions have been
executed for all systems, a new set is selected for each
system, and the process repeats.
 
State transitions are atomic; thus, for the execution of
Estelle specifications on uniprocessing systems, a process
scheduler may be invoked upon the completion of any state
transition to select the next transition (and hence process)
for execution.


\subsection[Interprocess Communication]{Interprocess
Communication}
\label{IPC}

Interprocess communication in Estelle is by message or
{\em interaction} passing on bidirectional FIFO message queues
({\em channels}) connected at their endpoints ({\em interaction
points}). As the process hierarchy may be an arbitrary
structure, the interconnections between processes may span
several levels of the hierarchy. Estelle does not allow the
direct connection of any one interaction point to any other
interaction point; instead, a connection between two
processes must be established by a common parent of both
processes (which may in fact be one of the processes
involved). The Estelle {\bf attach} statement then allows for the
extension of the connection down the process hierarchy as
necessary. It is important to distinguish between {\bf attach}
and {\bf connect}; {\bf connect} is used to establish connections while
{\bf attach} is used to extend those connections down through the
process hierarchy.

Each interaction point may be used to send interactions,
and is also associated with a FIFO queue (which may be
shared by other interaction points of the same module) from
which it can receive interactions. The set of interactions
which may validly be sent and received through an
interaction point are determined by the interaction point's
$<${\em channel-defn}$>$.

\begin{tabbing}
mm\=mmmmmmmmmmn\=mm\=mm\=\+\kill
$<${\em channel-defn}$>$\>=\>$<${\em chan-heading}$><${\em chan-block}$>$\\
\mbox{}\\
$<${\em chan-heading}$>$\>=\>{\bf channel} $<${\em chan-IDENT}$>$ {\bf (} $<${\em role-list}$>$
{\bf ) ;}\\
\mbox{}\\
$<${\em role-list}$>$\>=\>$<${\em role-IDENT}$>$ {\bf ,} $<${\em role-IDENT}$>$\\
\mbox{}\\
$<${\em chan-block}$>$\>=\>+\{ $<${\em interaction-group}$>$ \}\\
\mbox{}\\
$<${\em interaction-group}$>$\>=\>{\bf by} $<${\em role-IDENT}$>$ [ {\bf ,} $<${\em role-IDENT}$>$ ] {\bf
:}\\
\>\>\>+\{ $<${\em interaction-defn}$>$ \}\\
\mbox{}\\
$<${\em interaction-defn}$>$\>=\>$<${\em interaction-IDENT}$>$\\
\>\>\>[ {\bf (} $<${\em VALUE-PARAM-SPECIFICATION}$>$\\
\>\>\>\{ {\bf ;} $<${\em VALUE-PARAM-SPECIFICATION}$>$ \} {\bf )} ] {\bf ;}\\
\end{tabbing}

As can be seen, an interaction can have a number of value
parameters; none of these may be of a pointer-containing
type. For a given interaction point, a module assumes a
role declared by a $<${\em role-IDENT}$>$. A module may send
interactions associated with its assumed role; a module that
assumes the opposite role may receive interactions
associated with the first role.
 
An interaction point is an abstract bidirectional
interface through which a module may send or receive
interactions. Each interaction point has three attributes:

\begin{itemize}
\item the channel
\item the role
\item the queueing discipline
\end{itemize}

Any interaction associated with the role may be sent
through the interaction point; any interaction associated
with the opposite role may be received through the
interaction point.
 
The queueing discipline determines whether the queue
associated with that interaction point may be shared with
other interaction points of that module instance or not
(common or individual).

\begin{tabbing}
mm\=mmmmmmmmmmm\=mm\=mmmm\=\+\kill
$<${\em ip-decls}$>$\>=\>{\bf ip} +\{ $<${\em ip-decl}$>$ {\bf ;} \}\\
\mbox{}\\
$<${\em ip-decl}$>$\>=\>$<${\em IDENT-LIST}$>$ {\bf :} $<${\em ip-type}$>$\\
\>\>$|<${\em IDENT-LIST}$>$ {\bf : array [} $<${\em index-type-list}$>$ {\bf
]}\\
\>\>\>{\bf of} $<${\em ip-type}$>$\\
\mbox{}\\
$<${\em ip-type}$>$\>=\>$<${\em chan-IDENT}$>$ {\bf (}$<${\em role-IDENT}$>${\bf )}
[$<${\em q-discipline}$>$]\\
\mbox{}\\
$<${\em index-type-list}$>$\>=\>$<${\em INDEX-TYPE}$>$ \{ {\bf ,} $<${\em INDEX-TYPE}$>$
\}\\
\end{tabbing}

Interaction points may be referenced only by the
operations {\bf connect}, {\bf attach}, {\bf disconnect}, {\bf
detach}, the {\bf when}
clause of a transition, and in the {\bf output} statement.
Internal interaction points, which are defined in the $<${\em decl-
part}$>$ of a $<${\em body-defn}$>$, may only be bound and unbound using
{\bf connect} and {\bf disconnect}.

\begin{tabbing}
mm\=mmmmmmmmmmmm\=mm\=mmmm\=\+\kill
$<${\em connect-stmnt}$>$\>=\>{\bf connect} $<${\em connect-ip-ref}$>$ {\bf to} $<${\em
connect-ip-ref}$>$\\
\mbox{}\\
$<${\em connect-ip-ref}$>$\>=\>$<${\em internal-ip-ref}$>|<${\em
child-extern-ip-ref}$>$\\
\mbox{}\\
$<${\em disconnect-stmt}$>$\>=\>{\bf disconnect} ($<${\em connect-ip-ref}$>|<${\em
module-var}$>$)\\
\mbox{}\\
$<${\em attach-stmnt}$>$\>=\>{\bf attach} $<${\em external-ip-ref}$>$
{\bf to} $<${\em child-extern-ip-ref}$>$\\
\mbox{}\\
$<${\em detach-stmnt}$>$\>=\>{\bf detach} ($<${\em external-ip-ref}$>$
$|<${\em child-extern-ip-ref}$>$)\\
\mbox{}\\
$<${\em internal-ip-ref}$>$\>=\>$<${\em ip-ref}$>$\\
\mbox{}\\
$<${\em external-ip-ref}$>$\>=\>$<${\em ip-ref}$>$\\
\mbox{}\\
$<${\em child-extern-ip-ref}$>$\>=\>$<${\em module-variable}$>${\bf .}$<${\em
external-ip-ref}$>$\\
\mbox{}\\
$<${\em ip-ref}$>$\>=\>$<${\em ip-IDENT}$>$ [ {\bf [} $<${\em
INDEX-EXPR}$>$\\
\>\>\>\{ {\bf ,} $<${\em INDEX-EXPR}$>$ \} {\bf ]} ]\\
\end{tabbing}

The two interaction points referenced must be declared
with identical channel identifiers, and opposite role
identifiers for {\bf connect} or identical role identifiers for
attach. After execution of the {\bf connect} statement, any
interactions output through an interaction point are
received at the other (or its descendants if that
interaction point is attached). A {\bf connect} statement may
connect internal interaction points and child external
interaction points together in any combination.
 
The {\bf disconnect} statement causes the specified interaction
point, or all connected external interaction points of the
specified child module instance to be unbound. The
interactions for the interaction point(s) still remain in
the queue, and may still be processed.
 
The {\bf attach} statement attaches two interaction points, the
first of which may be currently bound and the second of
which may later be bound by an action of the child. An
interaction point at the end of a sequence of bound
interaction points is a {\em connection endpoint}. {\bf output}
statements at connection endpoints always cause their
interactions to be queued at the interaction point bound to
the opposite connection endpoint.
 
When the {\bf attach} statement is executed, interactions
present in the external interaction point queue which came
through this queue are removed, and appended to the queue of
the external interaction point of the lowest level
descendant module instance which attached to the child
external interaction point. The interactions for two
interaction points whose queuing option is individual queue
within a parent module may be combined into a common queue
of a child through {\bf attach} operations.
 
The {\bf detach} statement causes the specified interaction
point and the one (and only one) to which it has been
attached to be unbound. Any interactions queued at a
connection endpoint at a lower level than the current level
which were sent via the specified interaction point are
appended onto the interaction point's queue before the
interaction points are detached. A variant of this is called
simple detach; in this case no interactions are moved. The
terminate statement uses simple detaches rather than the
conventional detach.

\begin{tabbing}
mm\=mmmmmmmmnn\=mm\=mmmm\=\+\kill
$<${\em output-stmt}$>$\>=\>{\bf output} $<${\em interaction-ref}$>$[$<${\em
ACTUAL-PARAM-LIST}$>$]\\
\mbox{}\\
$<${\em interaction-ref}$>$\>=\>$<${\em ip-ref}$>$ {\bf .} $<${\em
interaction-IDENT}$>$\\
\end{tabbing}

The {\bf output} statement results in the interaction (plus its
parameters, if any) to be appended onto the queue assigned
to the (other) connection endpoint. Outputs made through an
external interaction point may be observed in the target
queue only after the whole issuing transition is completed,
due to the atomicity of transitions.

Interactions are received and removed from queues by
executing transitions with {\bf when} clauses. It may be necessary
to examine a queue without dequeuing an interaction, if the
transition containing the {\bf when} clause also contains a
{\bf provided} clause which refers to one or more of the
interaction paramaters. In this case, the interaction at the
head of the queue (assuming the {\bf when} clause is satisfied)
must first be examined to ensure that the provided clause is
also satisfied, before the interaction can actually be
dequeued.


