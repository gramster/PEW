\documentstyle [12pt,A4,lists]{article}
\begin{document}
\newpage
\thispagestyle{empty}
\setcounter{page}{0}
\vskip 8em
\begin {center}
{\LARGE The Estelle {\em PEW} User Manual}
\vskip 1em
{\large for Version 2.1}
\vskip 3em
{\large by Graham Wheeler}
\vskip 3em
{\large (c) November 1989-1992}
\vskip 1em
{\large Graham Wheeler}
\vskip 3em
{\large Data Network Architectures Laboratory}
\vskip 1em
{\large Department of Computer Science}
\vskip 1em
{\large University of Cape Town}
\vskip 1em
{\large Private Bag, RONDEBOSCH}
\vskip 1em
{\large 7700}
\vskip 1em
{\large South Africa}
\vskip 1em
{\large Telephone +27.21.650.2662}
\vskip 1em
{\large FAX +27.21.650.3726}
\vskip 1em
{\large E-Mail gram@cs.uct.ac.za or gram@gramix.aztec.co.za}
\end{center}
\newpage
\tableofcontents
\newpage
\pagestyle{plain}
\section[Preface]{Preface}

This manual is a revision of the Preliminary Version of the {\em PEW}
user manual which was released with Version 1.0 of the {\em PEW}.
It has been prepared for the release of Version 2.0 of the {\em PEW}.
In order to highlight the differences between the versions 1.0,
1.1, 1.2 and 2.0, the original introduction has been left unmodified.
Instead, supplementary introductions have been added, for each of the
successive versions, which summarise changes that have been made.
If you are already familiar with the {\em PEW}, you may read these
supplementary introductions immediately; for
newcomers, they will be more understandable once you have read the
rest of the manual and worked with the {\em PEW}.

For those who have used the {\em PEW} before, I hope you like the
changes that have been made, and find them useful.

\newpage
\section[Introduction to Version 1.0]{Introduction to Version 1.0}

The Estelle  {\em PEW} ({\em P}rotocol {\em E}ngineering {\em
W}orkstation) is  a tool  for the  development,  testing  and
simulation of  protocol specifications written using the Estelle
(Extended  State   Transition   Language)   Formal   Description
Technique (FDT).  The PEW  provides  an  integrated  environment
complete with  screen editor,  compiler,  linker,  and  
debugger/interpreter  all  in  a  single  package. 

The Estelle language was developed by the International Organisation
for Standardisation (ISO) as one of several techniques for protocol
specification. The {\em PEW} conforms closely to the Draft
International Standard DIS 9074, promulgated in July 1987. The major
restriction of the {\em PEW}'s implementation is the lack of a real
(that is, floating point) data type.

This document is the first version of the Estelle {\em PEW} User
Manual. It describes the first general release of the
{\em PEW}, which is Version 1.0. All previous releases were internal
to the Department of Computer Science of the University of Cape Town.

The {\em PEW} really began in 1987, when Jaque van Dijk, under the
supervision of Professor Pieter Kritzinger, developed a compiler for
Estelle. This compiler was completed in late 1988, at which
point Graham Wheeler took over the project. Inspired by the `Integrated
Environment' language development systems being marketed by the
Borland company (in particular, Borland's TurboC, which was used as
the development language for the {\em PEW}), Graham decided that
rather than just developing a stand-alone interpreter for Estelle, an
integrated development environment would be built.

The first component to be developed was the editor, which was written
during the early part of 1989. Once this was completed, Graham
attempted to integrate Jaque's compiler into the editor. The
obstacles to this proved insurmountable: the compiler had been
written in Pascal, and was also sufficiently unstable for it to be
executed indirectly from the editor. In April 1989, Graham wrote a
Pascal to C translator, and by May had the 6 500 lines of Pascal
translated to just under 6 000 lines of C. Over the course of 1989
the compiler was refined, large parts of it were rewritten, and many
bugs were removed. The final product is a stable compiler which is
20\% smaller (the executable code is 40\% smaller), and which compiles
Estelle code at about 5 000 lines per minute (about three times
faster than the original) - a true demonstration of the power of C!

Hand in hand with the redevelopment of the compiler went the
development of the interpreter. The compiler generates a (mostly)
low-level code called E-Code, based on Pascal's P-Code. The demands
of interpreting Estelle led to several rethinks of aspects of the
E-Code, which in turn led back to changes in the compiler. The two
components were thus inseperable in design, although they were kept
as seperate programs until July 1989 when the first integrated
version was produced. At this stage the debugger was primarily a
low-level debugger, allowing examination of the E-Code, stepping
through E-Code, examination of process's internal stacks, and other
details which have subsequently been hidden (but not removed).

Version 0.6 of the {\em PEW} was released internally within the 
Department of Computer Science at U.C.T. in August, which provided
valuable feedback about the system. Many thanks to the
senior students in the Department who unwittingly did the Beta
testing!

The current version of the {\em PEW} incorporates a new user
interface called the {\em Process Browser}. This is a multi-windowed
environment which allows the user to constantly view details of the
current executing process, or to examine other processes freely. This
environment makes it easy to set breakpoints, view connection
endpoints of channels, delete interactions from queues, examine the
Estelle source code, view statistics about the firing of transitions
and use of channels, and so on. Context-sensitive help is available
at all times. We believe this user-interface to be one of the most
exciting features of the {\em PEW}.

The fact that the {\em PEW} gathers execution statistics as it
operates makes it very easy to use as an execution profiling tool for
protocol specifications. For example, a user may calculate the
throughput and error rate of a protocol under various loads very
quickly and easily, without having to add explicit code to the
Estelle specification to gather statistics. The {\em PEW} also
detects and reports deadlock when it occurs, and can tell when there are
interactions queued which can never be dealt with. It thus aids the
correctness testing of protocol specifications.

This version of the {\em PEW} was demonstrated at Forte '89.
The {\em PEW} continues to be developed.
The next version is scheduled for
the end of 1990; we hope that we will be able to largely eliminate
the need for the user to perform multiple executions in order to
evaluate performance under different conditions. Our aim is to
incorporate an analytical tool into the {\em PEW} which can generate
and solve queueing models of protocol specifications.

We hope that you enjoy using the {\em PEW}. If you have any comments
or suggestions to make, or bugs to report, please make use of the
evaluation form included in the Appendices and return it to us.
         
\section[Introduction to Version 1.1]{Introduction to Version 1.1}

Version 1.1 of the {\em PEW} was released in February 1990. It
contains a number of changes, which are summarised below:
\begin{itemize}
\item A number of bugs have been fixed (of course).
\item The {\em F10} key is now used to quit the interpreter/debugger,
and the two different execution modes ({\em Animate} and {\em
Execute}) have been combined into a single mode serving both
purposes, assigned to the {\em F9} key.
\item Three new functions have been added: {\tt QLENGTH}, which
returns the length of a reception queue when applied to an
interaction point; {\tt FIRECOUNT}, which returns the number of
times the containing transition has been fired, and {\tt GLOBALTIME},
which returns the current (simulation) time.
\item The arrow on the left of the screen indicating which source line 
is being executed changes to an `s' when the scheduler is running.
\item Output procedures WRITE and WRITELN also write to the log file
if logging is on.
\item The destination of output from WRITE and WRITELN can be
controlled using a compiler directive.
\item The pseudo-random number generator can now be seeded by the user, to
replicate executions.
\item Semaphores are used to prevent the interpreter being executed
during critical sections; previously infinite recursions were
possible if the interpreter tried to dump statistics to the log
file when the process tree data structures were in a fluid state.
\item The process browser windows can now be changed in size (to a
limited extent) using {\em Shift-F1} and {\em Shift-F2}.
\item A low-level E-code display can be invoked from the process
browser with Alt-F10; this is of relevance only to those who are
interested in the internal implementation of the {\em PEW}.
\end{itemize}

\section[Introduction to Version 1.2]{Introduction to Version 1.2}

In March 1990, Version 1.2 of the {\em PEW} was completed. The
enhancements made in this version include:
\begin{itemize}
\item Some reachability and deadlock analysis of state transitions and
interprocess communication has been added.
\item As a result of the previous change, the compiler can also now
issue non-fatal warnings.
\item Pressing {\em F1} or {\em Alt-H} when an error or warning
window is displayed causes help for that error message to be
displayed. The help is contained in a plain text file {\tt ERROR.HLP}, 
which may be modified.
\item An {\em ANALYZE} menu has been added to the Editor/Compiler.
This menu allows an independent variable to be specified, together
with a range through which it should be varied. The {\em PEW} can
then automatically recompile and reexecute the specification for each
value of the independent variable (traeting it as a top-level
{\tt CONSTANT} definition). The user may specify expressions
to be evaluated and tabulated as a result of this process. This makes
it easy to produce tables of, for example, throughput and error rate
of a protocol against timeout.
\item {\tt ANY} constants are now supported.
\item The {\tt TERMINATE} statement, a new addition to Estelle, 
has been implemented.
\item The syntax of transitions now agrees fully with the standard, and
nested transitions are now supported.
\item All forms of the {\tt DETACH} statement are now implemented.
\item {\tt DELAY} clauses can now have either uniform, Poisson or
exponential distributions, selected by a compiler directive.
\item An environment variable, {\tt PEWPATH}, can be set to allow the {\em
PEW} to find its help files even if run from a different directory.
\end{itemize}

Thus, apart from the lack of a {\tt REAL} type, the only areas where
the {\em PEW} is still deficient from the standard is that \dots
types are still not allowed (and never will be, while the {\em PEW}
remains execution-oriented); {\tt ANY} constants are handled by
`cheating' somewhat), and the {\tt ANY} clause is not supported, but
will be soon (as soon as I can figure out an elegant way of doing it,
instead of the hack methods that I've thought of so far\dots).

\section[Introduction to Version 2.1]{Introduction to Version 2.1}

Version 2.0 of the {\em PEW} was completed in August 1991, with
additional changes since then bringing the latest version (May 1992)
to 2.1. The number of additions is considerable, and includes:

\begin{itemize}
\item Further compiler directives giving increased control over
inter-process communication, delays, and I/O;
\item The system has been split into stand-alone versions of the
compiler and interpreter, as well as the integrated version. The
stand-alone components can handle larger specifications;
\item The stand-alone interpreter can perform a `transition sequence
analysis' of some systems, and produce a graph specifying all
possible sequences of transitions under different probabilities of
message loss and delay values;
\item This graph can be checked for reachability by a separate
reachability checker, which can detect problems such as livelock and
deadlock;
\item The graph can also be processed by an additional utility which
can predict transition throughputs for different parameter values
(that is, {\tt OUTPUT} loss and {\tt DELAY} values).
\item The statistics output by the {\em PEW} are, in some cases, more
accurate, and are more comprehensive. We have used these statistics
to build performance models based on imbedded Markov chains. A
utility for solving such models is also included.
\item A trace of interaction outputs between the various processes
can be produced by turning on an option in the Options menu of the
{\em PEW}. This trace is written to a text file named {\tt TRACE}.
\end{itemize}

Unfortunately, the {\tt ANY} clause is still not supported. The {\em
PEW} has been used primarily with specifications of point-to-point
protocols, so this clause has not been needed.

Many of the additions were made as part of my Ph.D. thesis
which examined the problem of performance analysis
and prediction from formal specifications (specifically Estelle
specifications). A modelling method based on imbedded Markov chains
was employed. Tools to solve these models, and generate them
automatically in some cases, are also included in this distribution.
These tools are included in three separate {\tt .ZIP} files,
together with basic documentation.

\newpage
\section[Release Contents and Installation]{Release Contents and
Installation}

The {\em PEW} Version 2.1 distribution includes contains the files:

\begin {itemize}
\item {\tt pew.exe} -- the editor, compiler and interpreter
\item {\tt pew\_help.txt} -- the {\em PEW} Help Text file
\item {\tt pew\_help.idx} -- the {\em PEW} Help Index file
\item {\tt error.tbl} -- the {\em PEW} error and warning message table
\item {\tt error.hlp} -- the {\em PEW} error and warning help file
\item {\tt template} -- the Estelle language template definition file
\item {\tt langhelp} -- the Estelle language help file
\item {\tt pewdemo.dat} -- a script file containing commands treated
as keypresses by the {\em PEW}, for a quick demonstration
\item {\tt demo.bat} -- a DOS batch file to execute the {\em PEW} demo
\item {\tt abs.est} -- a specification of the Alternating Bit protocol,
used by the demo
\item {\tt ec.exe} -- the stand-alone compiler
\item {\tt ei.exe} -- the stand-alone interpreter
\item {\tt reach.zip} -- the reachablity checker, with example
\item {\tt predict.zip} -- the performance predicter, with example 
\item {\tt solve.zip} -- the model solver, with example
\item {\tt epretty.exe} -- an e-code pretty-printer
\item {\tt userman.tex} - the \LaTeX manual source
\end{itemize}

Additionally, the disk contains a zip file called {\tt test.zip},
which contains a number of small Estelle specifications that have
been used to test different aspects of the {\em PEW}.

These files should all be copied into the directory in which you
intend developing  your Estelle  specifications. If  you run the
{\em PEW}  from  a directory  other than that in which it is installed,
you should set the environment variable {\tt PEWPATH} to contain the drive
and directory path where the {\em PEW} and its related files can be
found. Consult your DOS manual for details on setting environment
variables if you are unfamiliar with this technique.

A program called {\tt unzip.exe} and a batch file called {\tt
install.bat} is bundled with the {\em PEW} distribution on floppy
disk. For {\tt ftp} distribution, the files are all in one zipfile,
so it is assumed you already have {\tt unzip} or its equivalent.

\newpage
\section[Using the {\em PEW}]{Using the {\em PEW}}

In order  to use  the {\em PEW},  a  specification  must  first  be
entered, using  the {\em PEW} editor or  some other  editor. The {\em
PEW} compiler must  then be  used to compile the specification. If an
error occurs  during compilation,  compilation stops and control
is returned  to the  editor. The  cursor is  positioned  at  the
offending location and the error is reported on the message line
at the bottom of the screen. When the error occurs, it is also
reported in an error window, which can be cleared by pressing any
key. An exception to this is pressing {\em F1} or {\em Alt-H}; these
keys will cause the {\em PEW} to look up the help text associated
with the error or warning, and display it.

Once a  specification has  been successfully compiled, it may be
executed using  the interpreter. Each of these components will be
discussed in turn in the following sections of the manual.

The file DEMO.BAT will cause the {\em PEW} to read `keystrokes' from
the file PEWDEMO.DAT, and give a short demonstration of some of its
features.

\newpage
\section[The {\em PEW} Editor]{The {\em PEW} Editor}

Upon invoking  the {\em PEW}, which you do by simply typing {\tt PEW},
you are shown a welcome message window.
Pressing any  key will  remove this  window and place you in the
{\em PEW} editor. The screen consists of a main menu bar running along
the top,  an edit  window  occupying  the  bulk  of  the  screen
(including an  information line  at the top), and a message line
at the bottom of the screen.

If you  started the {\em PEW} without a file argument, it will attempt
to open  a file called {\tt NONAME.EST} and load it. Otherwise it will
attempt to  load the  file that you specified. For example, to start
the {\em PEW} and load the file {\tt ABS.EST}, you can use the
command {\tt PEW ABS.EST}. If the {\em PEW} cannot load
the file you specified (or {\tt NONAME.EST} if you didn't specify any
file), the editor buffer displayed in the editor window will be
empty, but will be associated with the corresponding name.

It is possible to save your editor configuration explicitly at any
time, or automatically when you exit the {\em PEW}. If you do this,
and you invoke the {\em PEW} without a file argument, it will attempt
to load the file that was loaded when the configuration was last
saved. The editor configuration consists of the key assignments,
keystroke macros, option menu settings, and filenames.

Once in the editor, you may simply type text. Pressing {\em F1} or {\em
Alt-h}
at any stage
will give  you context-sensitive  help. Holding down the Control
({\em Ctrl}) key  and pressing  the capitalised  letter of a main menu
option will  result in  that option  being invoked from the main
menu. This may result in a submenu popping up.  For example, 
{\em Ctrl-F} will invoke the {\em File Menu}.

When you are in a menu, you may move around with the cursor keys
or by  pressing the  letter corresponding to a menu option (for
example, {\em L} for the {\em Load File} command in  the File  Submenu).
Pressing {\em ENTER} will select
that menu  option. In some cases this will result in a prompt on
the message  line (for example, {\em Load File} will prompt you for a file name).
Sometimes a  default value will appear after the prompt. You may
edit this  default or your own input using the cursor keys, Del,
Backspace, Ins  (to toggle insert/overstrike  mode) and  ESC (clear
value). When you are satisfied press {\em ENTER}. To leave a menu
without selecting an option use {\em ESC}.

We will now consider the various menus and commands available from
within the Editor.

\subsection[The Main Menu]{The Main Menu}

The main menu is displayed on the top line of the screen. It is
accessed from the editor by holding down the {\em Ctrl} key and
pressing the uppercase letter corresponding to the menu item
required. If one of the two menus ({\em File} or {\em Options}) is
invoked, and subsequently exited by pressing {\em ESC}, you will
remain on the main menu but the menu will close. You can then move
around the menu with the left and right cursor keys, and press {\em
ENTER} to select a menu item. To get back to editing, select the
{\em Edit} item. The main menu items are:

\begin{describe}{xxxxxxxx}
\item[\em Ctrl-F]Invoke the File Submenu (see
Section~\ref{filemenu}).
\item[\em Ctrl-E]Resume editing.
\item[\em Ctrl-C]Compile the Editor File.
\item[\em Ctrl-X]eXecute the Editor File, compiling it first if necessary.
\item[\em Ctrl-O]Invoke the Options Submenu (see
Section~\ref{optionmenu}).
\item[\em Ctrl-A]Invoke the Analyze Submenu (see
Section~\ref{analmenu}).
\end{describe}

\subsection[The File Submenu]{The File Submenu}

The File submenu contains commands primarily pertaining to the
manipulation of files, such as loading and saving files to and from
the editor. The items available are:

\label{filemenu}
\begin{describe}{xxxxxxxxxxxxx}
\item[\em Load File]Specify a file to load from disk into the editor
buffer.
\item[\em Load Log]Load the execution summary file `ESTELLE.LOG' from
disk. See Section~\ref{logfile} for details.
\item[\em Save]Save the edit buffer to disk using the current {\em Save
Name}.
\item[\em Change Name]Change the current {\em Save Name}.
\item[\em New]Clear the Editor buffer.
\item[\em Goto Line]Move the cursor to a specified line in the edit
buffer.
\item[\em OS Shell]Run a DOS shell. Type "exit" to return from DOS to
the {\em PEW}.
\item[\em Quit]Quit to DOS. You may also use {\em Alt-Z} to quit.
\item[\em Read Scrap]Read a disk file into the scrap buffer. The scrap
buffer is used for `cut and paste' editor operations.
\item[\em Write Scrap]Write the scrap buffer contents to a disk file.
\end{describe}

\subsection[The Compile and Execute Commands]{The Compile and Execute
Commands}

These two main menu options allow you to compile you Estelle program,
and invoke the interpreter. The {\em Compile} command will simply
attempt to compile your specification. If a compilation error
occurs, this will be reported in a pop-up error window. You may press
any key to remove this window (other than {\em F1} or {\em Alt-H},
which will cause help pertaining to the error to be displayed);
the error message will then be
shown on the message line and the cursor positioned where the
compiler encountered the error.

Error messages contain a text message, a line number, and an {\em
error code}. Error codes consist of an `E' followed by a three-digit
number. These error codes are used internally to identify the place
in the {\em PEW} where the error occurred, and can be ignored by the
user.

The {\em Execute} command will only compile the program if it has not
already been compiled. If compilation is unsuccessful, then you may
proceed as if you had just selected {\em Compile}. If compilation is
successful, after pressing a key to clear the Compile window (if
applicable) you will be placed in the {\em PEW execution
environment}. This will be dealt with in detail later.

\subsection[The Options Submenu]{The Options Submenu}
\label{optionmenu}

This menu contains various options which may be set according to 
your preferences for the editor; for example, how lines that are too
long to be displayed on a single screen line are to be treated. It
also is where you enter search and replace text strings for these
editor commands. The available items are:

\begin{describe}{xxxxxxxxxxxxxxxxxx}
\item[\em Tabsize]Specify the default tab stop setting. The {\em PEW}
editor handles tabs in a rather primitive fashion. Whenever the TAB
key is pressed, or a file is loaded, all tabs are expanded to spaces
using the current tabsize setting. When you save a file, this process
may be reversed depending on the {\em Fill Character} option (see
below). Thus while using the editor, all whitespace is effectively
made up of space characters, not tabs. To make this slightly less
unwieldy, the backspace key can be used with a Shift key to delete
all whitespace up to the previous tab stop.
\item [\em Backup Files]If this option is on, the {\em PEW} will
rename the old copy of a file before saving a new copy. The old
backup copy will have the same name but a `.BAK' extension.
\item[\em Fill Character]If this is set to Tabs, then spaces will be
compressed into tabs wherever possible when a file is saved to disk,
using the current {\em Tabsize} setting. See {\em Tabsize} for more
details.
\item[\em Wrap Mode]Switch between line wrap mode or horizontal
scroll mode. This affects the treament of long lines. When wrap mode
is on, long lines are `wrapped' over when necessary onto the next
screen line. When wrap mode is off, long lines are truncated at the
left and right edit window edges, and the entire display is scrolled
left and right as necessary.
\item[\em Left Search]This lets you specify a set of characters to be
used by the character set matcher. Whenever {\em Ctrl-PgUp} is
pressed, the cursor will be moved backwards to the first character in the edit
buffer that matches any character from the specified set. This can be
used as a simple parenthesis matching facility.
\item[\em Right Search]This is equivalent to the {\em Left Search}
option described above, except that the cursor is moved forward in
the file rather than backward.
\item[\em Search Text]Allows you to specify the current search text for the search
and translate commands.
\item[\em Translate Text]Allows you to specify the current
replacement text used by the translate command.
\item[\em Case Sensitivity]Lets you toggle case sensitivity on and off
for use in searching and translating.
\item[\em Identifiers Only]Lets you toggle identifier matching on and off
for use in searching and translating. When this option is on, the
search text must correspond to text in the edit buffer which is
surrounded by whitespace or punctuation only. This is useful is you
wish to, for example, change all occurences of the identifier
`alternating\_bit' to `AB', but you don't want to change any
occurences of the identifier `alternating\_bit\_body' in the process.
\item[\em Autosave Config]When this option is on, the editor
configuration is saved automatically upon exit. The configuration is
saved in the file {\em PEW.CFG}.
\item[\em Save Config]Save the configuration immediately. The editor
configuration consists of the current file name and position,
option menu settings, and key assignments and macros.
\item[\em Random Seed]This option allows you to specify a seed for
the pseudo-random number generator (the seed is usually chosen
randomly).  It allows you to replicate an execution, even though
executions are, in principle, nondeterministic.
\item[\em Generate Trace]If this option is on, the {\em PEW} will
produce a file called {\tt trace} when the specification is executed.
This file shows the sequence of interactions that are exchanged
between processes over time.
\end{describe}

\subsection[Fixed (Non-Reassignable) Commands]{Fixed (Non-Reassignable)
Commands}

At present, all editor commands are non-reassignable; however,
all that is lacking is the ability to edit the default key
assignments. The commands listed in this section are those whose
key assignments are "hard-coded" into the editor, and cannot be
changed even when a key-assignment editor has been written.

\begin{describe}{xxxxxxxxxxxxxxx}
\item[$\uparrow$]Move the cursor up one line.
\item[$\downarrow$]Move the cursor down one line.
\item[$\leftarrow$]Move the cursor left one character.
\item[$\rightarrow$]Move the cursor right one character.
\item[\em Home]Move the cursor to the start of the current line.
\item[\em End]Move the cursor to the end of the current line.
\item[\em PgUp]Scroll the screen up.
\item[\em PgDn]Scroll the screen down.
\item[\em Del]Delete the character under the cursor.
\item[\em Backspace]Delete the character immediately before the
cursor.
\item[\em Shift-Backspace]Delete all whitespace characters up to the
previous tab stop setting.
\item[{\em F1} or {\em Alt-H}]Provide help.
\end{describe}

\subsection[Other Cursor Movement Commands]{Other Cursor Movement Commands}

This section lists the remaining cursor movement commands that are
available. Remember, you can use the {\em Goto Line} command on the
File menu to position the cursor on a specific line.

\begin{describe}{xxxxxxxxxx}
\item[\em Ctrl-$\leftarrow$]Move cursor left one word.
\item[\em Ctrl-$\rightarrow$]Move cursor right one word.
\item[\em Ctrl-Home]Move cursor to the start of the file.
\item[\em Ctrl-End]Move cursor to the end of the file.
\item[\em Ctrl-P]Set a bookmark. Up to ten positions in the file can
be remembered with this facility, and recovered with the {\em Goto
last bookmark} command described below.
\item[\em Ctrl-G]Goto last bookmark. This command restores a
remembered cursor position that was saved with the {\em Set a
bookmark} command described above. Positions are restored in the
opposite order to which they are saved, and the bookmark list is
cyclic.
\end{describe}

\subsection[Block Commands]{Block Commands}

\begin{figure}
\vspace{15cm}
\caption{\label{normal-block} An example of a Normal block}
\end{figure}

This section describes commands used to cut and paste blocks of text.

Blocks can be cut and pasted in two forms: {\em normal} blocks and
{\em column} blocks. Normal blocks are simply contiguous character
sequences from the file, for example, all text from line 10 column 13
to line 17 column 2 inclusive.
Column blocks are columnar substrings
of line ranges, for example, columns 10 through 20 of lines 10
through 19 inclusive.
There are commands for marking blocks of either
type, and for pasting blocks of either type. The two types can be
combined, that is, a block can be cut as a normal block and pasted as
a column block, and vice-versa.

Blocks are delimited by firstly dropping a {\em mark} of the
appropriate type at the one extreme, and then moving to the other
extreme position. The block will be highlighted during this process.
The block may then be cut or copied to the scrap buffer,
from where it may be pasted.

It is also possible to save the contents of the scrap buffer to a
file, or load the contents of a file into the scrap buffer. These
operations were described in Section \ref{filemenu}.

\begin{describe}{xxxxxxxxxxxxxxxxxxx}
\item[\em Alt-M]Set a normal mark.
\item[\em Alt-C]Set a column mark.
\item[\em Alt-- (Alt-Minus)]Copy marked text to scrap buffer.
\item[\em Del]Delete marked text, copying to scrap buffer.
\item[\em Ins]Paste scrap buffer contents at cursor position as a
normal block.
\item[\em Alt-P]Paste scrap buffer contents at cursor position as a
column block.
\end{describe}


\subsection[Search/Translate Commands]{Search/Translate Commands}

The search/translate commands provide the means to loacte specific
text or characters, and to replace occurences of some text string
with a different text string. 
See Section \ref{optionmenu} for details on how to specify the search and
translate text patterns, and character match sets.

\begin{describe}{xxxxxxxxxx}
\item[\em Alt-S]Search forward from cursor position for search text.
\item[\em Ctrl-S]Search backward from cursor position for search
text.
\item[\em Alt-T]Translate text which matches the search text to the
specified translate text.
\item[\em Ctrl-PgUp]Perform left character match.
\item[\em Ctrl-PgDn]Perform right character match.
\end{describe}

\subsection[Miscellaneous Commands]{Miscellaneous Commands}

\begin{describe}{xxxxxxxx}
\item[\em Alt-Z]Quit the {\em PEW} and return to DOS.
\item[\em Alt-U]Undo all changes that have been made to the current line
since moving the cursor onto this line.
\item[\em Alt-R]Record a keystroke macro and assign to key. Keystroke
macros can be assigned to the keys Alt-0 through Alt-9. At present,
they may only be used in the editor, not in the execution
environment. After pressing Alt-R, you will be asked to press the key
to which you are assigning the macro. All keypresses will
be recorded until the macro key is pressed again. Subsequent presses
of the macro key will insert the macro characters into the keyboard
buffer, where they will behave as though they had been pressed.
\item[\em Alt-D]Delete the current line.
\item[\em Alt-F1]Call up language help for the Estelle keyword at the
current cursor position, if any.
\item[\em Ctrl-F1]Perform template expansion on the Estelle keyword
at the current cursor position, if any.
\end{describe}

The last  two commands make use of the 'template' and 'langhelp'
files. Both of these files may be
freely edited, allowing you to create your own help and template
details.  In  each  case,  the  word  at  which  the  cursor  is
positioned is  used as a lookup index into the appropriate file;
if an  entry is  found, it is displayed (in the case of help) or
it replaces the index word (in the case of template expansion).
The file formats consist of alphabetically sorted entries, where each
entry begins with a index line consisting of an `@' character
followed by the string which is used as the match index, and then a
number of lines containing the help or replacement text. For the help
file, the tilde character `\~' can be used to indicate the start of a
new line. You must maintain the alphabetical ordering of the
indices, as a binary search method is used on these files and they
are assumed to be sorted.

\subsection[The Analyze Submenu]{The Analyze Submenu}
\label{analmenu}

The Analyze menu provides you with a way to execute a specification a
number of times, varying a single parameter (called the {\em
independent variable}) through a range of values. Up to eight {\em
dependent expressions} can be entered using an expression editor, and
the values of these expressions will be tabulated in the {\em PEW}
log file.

The use of the Analyze menu is best illustrated by example. We shall
use the sample specification of the Alternating Bit protocol, to
calculate the throughput and error rate of this protocol as a
function of the timeout. We will vary the timeout from 3 seconds to 6
seconds, in steps of one second each. For each timeout, we will
excercise the protocol for 15 minutes of simulated time (9000
seconds).

The throughput and error rate of the protocol depend on the total
time, and the execution counts of two transitions: the {\em timeout
and retransmit} transition, and the {\em received acknowledgement}
transition. The first represents the number of unsuccessful
transmissions, while the second is the number of successful
transmissions. In the specification, these transitions are named {\em
Retrans} and {\em GotACK} respectively. This gives us the equations:

\begin{itemize}
\item $Thruput = GotACK / time$
\item $Err_Rate = Retrans / (Retrans+GotACK)$
\end{itemize}

We now have sufficient background to perform the analysis. The steps
involved are as follows:

\begin{itemize}
\item Load the {\tt abs.est} specification into the editor.
\item Comment out or delete the line in the constant definition
section which specifies {\tt Timeout=4}. This is necessary as the
analysis process will be defining this constant for us.
\item Press {\tt Ctrl-A} to invoke the analyse menu.
\item Move the highlight to {\em Independent Variable} and press {\em
ENTER}. You will be prompted to enter a variable name. Enter {\tt
Timeout}.
\item Move the highlight to, select and enter the values 3, 6 and 1
for the menu options {\em From Value}, {\em To Value} and {\em
Stepsize} respectively.
\item Select the {\em Edit Dependents} option. A new window will
appear on the screen; this is the {\em expression editor}. It works
similarly to the usual editor, except that the {\em ESC} key clears
the current line and the {\em RETURN} or {\em ENTER} key returns to
the menu. The cursor keys, {\em Ins}, {\em Del} and {\em Backspace}
all work as usual. Enter the expressions for throughput and error
rate as they appear above. The general rules for expressions are that
(i) an expression must have a unique identifier on the left hand of the
assignment, and must only refer to transition names and the
predefined identifier {\em time} on the right hand side, and (ii) an
expression may not be split over more than one line. Expressions are
only parsed after all the executions are complete, so it is often
wise to do a quick test run with a single value of the independent
variable. Badly parsed expressions, or those that involve division by
zero, have their results indicated as a string of hashes (\#\#\#\#). In
Version 1.2 of the {\em PEW} expressions may not refer to other
expressions, and are not saved when the {\em PEW} is quit.
\item Select the {\em Analyze} option. You will be prompted for an
execution timelimit. Enter 9000.
\end{itemize}

The {\em PEW} will now compile and execute the AB specification for
each value of {\tt Timeout}, tabulate the results of all named
transitions, and the results of all dependent expressions. We can
then examine the log file, and see how our timeout setting affected
the throughput (measured in frames per second) and error rate (a
proportion).

In the case of the AB specification, there are two instances of AB
protocol processes executing. The {\em PEW} keeps track of the number
of instances of a process as well, and uses the mean value of the
execution count when evaluating expressions. In other words, if the
first AB instance had 47 retransmissions, and the second had 51, the
{\em PEW} will report and use the value 49.

\newpage
\section[The Execution Environment]{The Execution Environment}
\label{execenv}

We now turn to the interpreter and its user-interface, which we refer
to as the {\em PEW Execution Environment}. This is a separate
environment to the editor environment in that files cannot be edited
in the execution environment, and the main menu bar is inaccessible,
with the exception of the {\em Edit} command which returns to the editor
environment. The latter can be done by pressing {\em Ctrl-E} as
before, or alternatively {\em Alt-Z}.

\subsection[The Implementation of Time]{The Implementation of Time}
Before we consider the user-interface to the execution environment,
it is appropriate to explain how process scheduling and the handling
of time has been implemented in the {\em PEW}. 
We distinguish between the {\em simulation time} and the {\em
scheduler iteration}.
Execution proceeds systematically iteration by iteration until
all possibilities  are exhausted  - only  then are {\tt DELAY} clauses
examined to  determine if  progress is possible in the future or
whether deadlock  has occurred.  If all  that is  preventing the {\em
PEW} from proceeding is a {\tt DELAY} clause, the {\em simulation
time} is
increased by the smallest delay found. This  process  continues
indefinitely  until deadlock occurs or the user terminates it.

{\tt DELAY} clauses are handled as follows: in the case where only
one argument is specified, that argument is used as the delay value.
Where two arguments are specified, a delay value between the two is
chosen (arbitrarily, according to the specified distribution)
as the actual delay. Within a specification, compiler directives can
be sprinkled specifying the delay distribution to be used. The
default is a uniform distribution, selected with the directive
\{\$U\}.
Also available are Poisson distributions, geometric distributions,
and an exponential distributions. Each of these requires an argument
specifying the expected value, and they have the forms \{\$P{\em val}\},
\{\$G{\em val}\} and \{\$E{\em val}\} respectively.

As time only progresses when {\tt DELAY} clauses are present, the
{\em PEW} differs from the Estelle standard, which specifies that
time must eventually advance. A specification with no {\tt DELAY} clauses will never
advance beyond time 1 in the {\em PEW}, as all transitions without
{\tt DELAY}s are
regarded as requiring no time to execute. It was felt that this was a
more appropriate implementation as it affords the user more control
over the handling of time. 

At each iteration, every  transition of every active process is
scrutinised to see if  it is  enabled. Subsequently  we iterate
through  each  active  process and execute   any   selected
transitions. The process thus consists of:

\begin {itemize}
\item Step 1:    Evaluate all transition clauses
\item Steps 2-n: Iterate recursively through each active process, executing a transition if
appropriate
\end{itemize}

\subsection[Controlling Execution]{Controlling Execution}

The first step described above is regarded as atomic and non-interruptable in the
{\em PEW}. It is only possible to step execution through transition
bodies, {\em not} their clauses. However, the {\em PEW} provides
tabular information about clause evaluation, so
that the results of this process can be examined. For controlling the
execution of a specification, several alternatives are available.

It is possible to single-step execution in the {\em PEW} on three
different levels: the level of Estelle statements, the level of
transition blocks, and the level of scheduler iterations. This is
achieved using the {\em F2}, {\em F3} and {\em F4} keys respectively.

Execution may also be {\em animated}, that is, execution may
proceed automatically with a user-specified delay between the
execution of each Estelle statement. To select this option, use {\em
F9}. You will be prompted to enter an inter-statement animate delay
in units of milliseconds. Entering a delay value of zero (the
default) specifies {\em Fast Execution}. This
is useful when you wish to quickly execute to a certain time or
breakpoint, or are only interested in the execution summary. Unlike
animation where full screen updating is constantly  performed,
during fast execution only the time display is updated. This leads to a great
speed improvement in execution.
After specifying the delay, you will
be asked to specify a time at which execution is halted and
control returned to the user. If you enter a value of zero (the
default), execution will continue until an interruption or
breakpoint.

Note that the timing routine used for animation delays is
ill-behaved on some hardware; if you select animation and nothing
seems to be happening, wait for about 15 seconds and the animation
should begin.

Both animation and fast execution may be interrupted at any time
by a keypress. When memory runs low, the {\em PEW} will not allow
fast execution to occur, but will always switch back into single-step
mode.

\subsection[The Process Browser]{The Process Browser}

Within the execution environment, the screen display changes
considerably from the editor environment.

The first thing to notice is that the edit window is still present,
but now only displays five lines of the Estelle source. Three new
windows have been added to the screen; these, together with the
message line at the bottom of the screen, constitute the {\em Process
Browser}. The windows are entitled {\em Children}, {\em Transitions},
and {\em Interaction Points}. The sizes of the new windows may be
altered by using {\em Shift-F1} and {\em Shift-F2}. The size of the
edit window is fixed.

At any stage, there is a single {\em current active process}.
During execution, this process is the same as the {\em currently
executing process}; however, during browsing, the user may make any
process the current active process without affecting the currently
executing process. The Process Browser always displays information
pertaining to the current active process, specifically its module
variables (in the {\em Children} window), its transitions (in the {\em
Transitions} window) and its interaction points (in the {\em
Interaction Points} window). The attribute, module variable name,
module body type and current state of the current active process are displayed on the
message line. A two character abbreviation is used for attributes,
namely:
\begin{itemize}
\item Un - Unattributed
\item SP - Systemprocess
\item SA - Systemactivity
\item Pr - Process
\item Ac - Activity
\end{itemize}
In the case of the specification itself, the name of the
specification is displayed as the module body type, while the keyword
`Specification' is displayed in place of the module variable name.

At any stage there is a {\em current active window}. This window
contains a highlight bar, which can be moved up and down with the
cursor keys. The highlight bar is used to select items (be they
transitions, module variables, source lines or interaction points)
upon which actions may be taken. Only non-empty windows may be
active. To switch from one window to another, the {\em F6} key is
used. Within each window, pressing {\em ENTER} results in some
action, usually the popping up of a menu. Each window will be
considered in turn.

\subsubsection[The {\em Source} Window]{The {\em Source} Window}

The source window is the original edit window. The name {\em Source}
window rather than {\em Edit} window is to emphasise the fact that
it is not possible to edit the file during execution.
This window allows the user to examine the
Estelle source code, and to track execution when single-stepping on
Estelle statements.

The source window indicates the currently executing source line with
an arrow on the left-hand border of the window. This arrow may or may
not be visible, depending on whether the executing line  is visible
within the window or not (if not, it is because the highlight bar has
been moved away from it; the highlight bar is always visible in the
active window and the window contents are scrolled if necessary to
allow this). When the scheduler is running, this arrow changes to an
`s'.

Pressing {\em ENTER} inside the source window invokes the {\em Line
Menu}. This menu has three commands: {\em Set Breakpoint}, {\em View
Breakpoints} and {\em Goto Line}.

The {\em Set Breakpoint} and {\em View Breakpoints} commands are
common to the other window menus as well, although the type of
breakpoint created by the {\em Set Breakpoint} command varies
depending on the window. For the source window, breakpoints are set
on Estelle source lines. Before each line is executed, a check is
made to see if a breakpoint has been set on that line, and if so,
this breakpoint is processed before the line is executed. The details
of breakpoint setting and the {\em View Breakpoints} command will be
dealt with in Section~\ref{brkpt-sec}.

The remaining command, {\em Goto Line}, is identical to the editor
version; you will be prompted to enter a line number, and the
highlight will subsequently be positioned at that line.

Remember, help is available at any stage by pressing {\em F1}.

\subsubsection[The {\em Children} Window]{The {\em Children} Window}

This window displays the module variables of the current active
process. It should be noted that the {\em PEW} gathers much of its
symbolic information during execution of a specification. This has
several implications, amongst them the fact that it only knows that a
process has module variables once it begins to execute that process,
and that it only knows the names of module variables once they have
been initialised. Uninitialised module variables are simply dsplayed
as having the name `Uninitialised'.

Pressing {\em ENTER} on a highlighted module variable will make the
 process associated with
that module variable the current active process. In this way it is
easy to move down the process hierarchy. To move up this hierarchy,
the {\em F5 Up Level} key is used; this activates the parent of the
currently active process. When a different process is activated, all 
the windows of the process browser will change to reflect the new
active process.

By using the {\em F5} key in conjunction with the {\em Children}
window, it is a simple matter to examine any process.

\subsubsection[The {\em Transitions} Window]{The {\em Transitions} Window}

This window shows the transitions associated with the current active
process. In each case, the priority is displayed (if there is one),
followed by the {\em from} and {\em to} states of the transition.
These are followed by details about the clause evaluation of each
transition, namely the state of the {\em Provided}, {\em When} and
{\em Delay} clauses, whether the transition was {\em Enabled}, and
whether it has been {\em Selected} for execution. Each of these is
identified by an appropriate letter, namely {\em W, P, D, E} and {\em S}
respectively. In each case the convention used is 0 for false, 1 for
true, and a -- for not applicable (in other words, no such clause
exists for that transition).

Furthermore, the transition window will indicate the
currently executing transition with an asterisk on its left. If the
current active process is not the current executing process, no
asterisk will be seen.

Pressing {\em ENTER} on a highlighted transition will invoke the {\em
Trans Menu}. This menu includes the {\em Set Breakpoint} and {\em
View Breakpoints} commands mentioned earlier, which are discussed
in Section~\ref{brkpt-sec}.

The remaining command in the {\em Trans Menu} is {\em View Stats}.
This command displays a table of statistics associated with the
currently highlighted transition. This table contains the fields:
\begin{describe}{xxxxxxxx}
\item[\em From]The {\em from} state of the transition;
\item[\em To]The {\em to} state of the transition;
\item[\em Enabled]A count of the number of times the transition has been enabled
for execution;
\item[\em Fired]A count  of the number of times the transition has been fired
or executed. This may differ from the previous figure, as being
enabled does not necessarily imply being selected for execution;
\item[\em First]The time at which the transition was first enabled;
\item[\em Last]The most recent time when the transition was enabled;
\item[\em Back]The mean time elapsing between this transition being
executed and the transition just preceding it being executed;
\item[\em Self]The mean time elapsing between successive occurences
of this transition;
\item[\em Forward]The mean time elapsing between this transition
being executed and the transition just following it being executed.
\end{describe}
In the case where multiple {\em from} states where specified in the
transition, the {\em from} state will be displayed as a number of
periods `...'.

\subsubsection[The {\em Interaction Points} Window]{The {\em Interaction
Points} Window}

This window shows the names, reception queue lengths and reception
queue contents of the interaction points of the current active
process. It is possible to scroll through a long queue by
highlighting its interaction point and using the left and right
cursor keys to move forwards and backwards respectively.
The displayed length will change to reflect the length
of the queue from the first {\em displayed} interaction to the end of
the queue.

By pressing the {\em DEL} key, the first {\em displayed} interaction
in the currently highlighted queue can be deleted. Confirmation of
the deletion will be requested before it is performed. This provides
a way to test how a specification handles lost interactions without
having to write Estelle code to randomly lose interactions.

Pressing {\em ENTER} invokes the {\em IP Menu} for the currently
highlighted interaction point. As for the {\em Line Menu} and {\em
Trans Menu}, the {\em Set Breakpoint} and {\em View Breakpoints}
commands are once again available. Two other commands are available:
{\em View Stats} displays a statistics table for the currently
highlighted inetraction point, while {\em View CEP} will find the
connection endpoint of this interaction point, and activate its
containing process, as well as highlighting the appropriate
interaction point. This latter command provides an easy way to
examine the statistics for interactions output through an interaction
point: the statistics and queue displayed are for reception only,
thus by viewing the reception statistics of the connection endpoint of
an interaction point, the transmission statistics and queue can be
seen. These are complicated in the case of common queues or complex
connection and disconnection sequences but these are rare in
practise.

The statistics table for an interaction points contains the fields:
\begin{describe}{xxxxxxxxxxxxxx}
\item[\em Name]The name of the interaction point;
\item[\em Total Traffic]The total number of interactions dequeued
through this IP due to the execution of {\tt WHEN} clauses;
\item[\em Mean Length]The average length of the reception queue of
this IP;
\item[\em Max Length]The longest length ever reached by the
reception queue of this IP;
\item[\em Mean Time]The average length of time spent by an
interaction in the reception queue before being dequeued;
\item[\em Max Time]The maximum length of time spent by an
interaction in the reception queue before being dequeued.
\end{describe}

\newpage
\section[Breakpoints]{Breakpoints}
\label{brkpt-sec}

The {\em PEW} allows breakpoints to be set on line numbers being
reached, transitions being enabled or executed, and interactions
being output through interaction points. The type of breakpoint
depends on which window is active when the breakpoint is set;
breakpoints are always set on the currently highlighted item (line,
transition or interaction point) in the current active window. For
transition breakpoints, the user must also specify whether the
breakpoint is on {\em enablement} or {\em execution} of the
transition.

Breakpoints may include {\em pass counts}, specifying a number of
times that the breakpoint conditions must hold before the breakpoint
{\em matures} or fires. The default pass count is zero.

Breakpoints may be {\em active} or {\em suspended}; suspended
breakpoints do nothing and are not checked, but may be reactivated
later explicitly by the user, or by the action of some other
breakpoint. The other action that may be taken by a breakpoint is to
dump specified information to the log file {\tt ESTELLE.LOG}.

After maturing, a breakpoint may be {\em reset} or {\em suspended}.
If it is reset, its pass count will be reset as well, so it will
behave as though it had just been set by the user for the first time.

It is also possible to specify  whether a breakpoint should return
control to the user or simply continue with execution. The latter is
useful when the breakpoint has an action such as reactivating a
different breakpoint or dumping information. Often a breakpoint will
have no action other than to return control to the user.

Breakpoints set on transitions and interaction points may be
constrained to affect only the process that was active when they were
set, or to hold across all peer processes with the same body type. In
the latter case, any such process which satisfies the breakpoint
conditions will cause the pass count to be decremented, and this pass
count is shared between all such processes. In other words, a
breakpoint with a pass count of five that is set on a single process
will mature after that process has satisfied the breakpoint
conditions five times, while one set on any process of that type will
mature after it has been satisfied five times in total by any of the
appropriate processes. Thus, breakpoints set on a single process give
a finer degree of control than those set on process types.

\begin{figure}
\vspace{15cm}
\caption {\label {brkpt-menu} The Breakpoint Menu}
\end{figure}
When the {\em Set Breakpoint} command is selected from one of the
window menus, the {\em Breakpoint Menu} is invoked (see Figure
\ref{brkpt-menu}).
The bottom few lines of this menu give details about
the breakpoint, namely the {\em breakpoint number}, the
item on which the breakpoint is being set, and the current active
process, if relevant.

The {\em Action} fields specify whether the
breakpoint should activate some other suspended breakpoint, and if
so, which one (specified by its breakpoint number), and 
whether information should be dumped to the log file.
The dumped information consists of any combination of transition and
interaction point information, corresponding to the window contents
and statistics tables. The information may be dumped for any of the
following cases:
\begin{itemize}
\item The current process
\item The current process and all its peers of the same body type
\item The current process and all its peers
\item All processes
\end {itemize}
The information to be dumped is selected with the {\em Dump Pane}
option, while the case is selected with the {\em Process} option.

The first few options in the {\em Breakpoint Menu} specify:

\begin{itemize}
\item {\em Pass Count} The number of times the breakpoint must mature
before any action will be taken
\item {\em Process} Whether the breakpoint applies to the current
process only, or to all peer processes with the same body type (this
is not applicable for line breakpoints)
\item {\em Transition} If the breakpoint is set on a transition, is
it for the enablement of the transition, or its execution?
\item {\em Control} Should control return to the user when the breakpoint
matures, or should execution continue?
\item {\em Reset} Should the breakpoint be reactivated after
maturing, or should it be suspended?
\item {\em Active} Is the breakpoint currently active or suspended?
\end {itemize}
Once all the relevant information has been selected, the {\em Ready}
command at the top of the menu should be executed. If you wish to
cancel setting the breakpoint before selecting {\em Ready}, press
ESC.

If the {\em View Breakpoints} command is selected from a window menu,
the {\em Breakpoint Browser} is invoked.
This browser allows the user to examine,
modify or delete 
breakpoints. Suspended breakpoints have parentheses about their
number. By pressing {\em Del} the user may delete
the currently highlighted or all breakpoints; by pressing {\em Enter}
the {\em Breakpoint Menu} is reinvoked for that breakpoint so that
it may be edited.

\newpage
\section[The Log File]{The Log File}
\label{logfile}
At the end of execution, the logfile contains a summary of queue
traffic and  transition firing  statistics as well as information
dumped as a result of breakpoint actions, and having the Log option
on. This latter option is toggled using the {\em F8} key and turns
full information dumping on for all processes at every scheduler
iteraction. This allows a detailed trace of the execution of a
specification to be captured. Be warned that such traces grow very
large very quickly!

The log file enables us to
determine the  probabilities of  transitions firing, and thus to
easily calculate throughput and error rates of protocols.

The log file also contains matrices of {\em transition sequence
execution counts} and {\em transition sequence delays}. These are the
number of times each transition in each process was followed by each
other transition in that process, and the mean delay in each case.
These figures are used for building Markov process models of the
long-term behaviour of the executing specifications when viewed as
stochastic processes.
At the end of the log, global matrices are printed out. The global
transition sequence count matrix includes {\em linkage counts}
between processes showing the relationships between transitions that
{\tt OUTPUT} interactions and the corresponding receiving {\tt WHEN}
transitions.

\newpage
\section[Technical Reference]{Technical Reference}

This section details the limitations of the {\em PEW}, and its error
messages. For details on the restrictions and extensions of this
implementation of Estelle, refer to the Appendix.

\subsection[System Limits]{System Limits}

\vspace{1cm}
{\large Hardware Requirements:}

\begin{itemize}
\item 8086 processor running MS-DOS or PC-DOS (i.e., IBM PC, PC-AT or
compatible), with at least 256kB of memory
\item An EGA/VGA or compatible display is preferable.
\end{itemize}

\vspace{1cm}
{\large Editor Limits:}

\begin{describe}{xxxxxxxxxxxxxxxxxxxxxxxxxxxxxxxxxxxxx}
\item[Maximum line length]255 characters
\item[Maximum character search set]15 characters
\item[Maximum number of place marks]10
\item[Maximum keystroke macro length]128 keystrokes
\end{describe}

\vspace{1cm}
{\large Compiler Limits:}

\begin{describe}{xxxxxxxxxxxxxxxxxxxxxxxxxxxxxxxxxxxxx}
\item[Maximum string constant length]80 characters
\item[Maximum number of case indices]256; all case values must fall
in the range of 0..255.
\item[Maximum allowed distinct identifiers]500
\item[Symbol Table Size]30 kB
\item[String Table Size]5000 bytes
\item[Code Buffer Size]10000 words
\item[Maximum identifiers in {\tt EXIST} expression]16 (this also applies
to the {\tt ALL} and {\tt FORONE} constructs).
\item[Maximum Set Size]64; all set values must fall in range 0..63.
\item[Maximum Scope Levels]16. {\tt WITH} statements and
others that introduce new identifiers all use one or more scope
levels.
\end{describe}
\vspace{1cm}
{\large Interpreter Limits:}
\begin{describe}{xxxxxxxxxxxxxxxxxxxxxxxxxxxxxxxxxxxxx}
\item[Maximum per-process heap size]1024 words (characters and
integers each use up one word).
\item[Maximum number of breakpoints]32
\end{describe}
\vspace{1cm}

\subsection[System and Editor Error Messages]{System and Editor Error Messages}


\begin{description}
\item[Memory allocation failure]\mbox{}\\
An attempt to allocate memory failed. This error often results in
the {\em PEW} aborting. If you have been editing a file, this will be
saved to a temporary file and a message to that effect will be
printed.

\item[Illegal argument passed to function {\em fn}]\mbox{}\\
This is an internal error. It is unlikely to ever occur.

\item[Cannot save screen image]\mbox{}\\
You attempted to run a DOS shell but the {\em PEW} could not
allocate sufficient memory to save the screen contents. In this
case, it is very unlikely that a shell could be executed successfully
anyway!

\item[Cannot find COMMAND.COM]\mbox{}\\
You attempted to run a DOS shell but the {\em PEW} could not find
your copy of {\tt command.com}.

\item[Insufficient memory]\mbox{}\\
An attempt to run a DOS shell failed due to lack of available
memory.

\item[Error while trying to execute COMMAND.COM: {\em n}]\mbox{}\\
An error occurred during an attempt to run a DOS shell. The number
corresponds to the DOS error code for whatever error occurred.

\item[Cannot open file: {\em fl}]\mbox{}\\
You attempted to load a nonexistent file into the editor.

\item[Illegal input: integer value required]\mbox{}\\
You executed some command which expected a number as input but
found something else instead.

\item[Illegal input: line number out of range]\mbox{}\\
You attempted to execute a {\em Goto Line} command to a line which
does not exist.

\item[Illegal tabstop value]\mbox{}\\
You entered an invalid tab stop value.

\item[Cannot open help index file]
\item[Cannot open help text file]\mbox{}\\
You are missing one or both of the {\em PEW} help files in your
PEWPATH directory, or current directory if the PEWPATH environment
variable is not being used. Check that you have these files, and that
you are in the same directory as them, or they are in the directory
specified by PEWPATH.

\item[Cannot record macro when already busy recording]\mbox{}\\
You attempted to begin recording a keystroke macro while one was
already busy being recorded. Keystroke macros cannot be nested.

\item[Cannot play back macro while recording]\mbox{}\\
You attempted to play back a keystroke macro while busy recording
one.

\item[Cannot assign macro to this key]\mbox{}\\
Keystroke macros can only be assigned to the keys {\em Alt-0}
through {\em Alt-9}.

\item[Macro exceeds maximum length ({\em n} keys)]\mbox{}\\
You exceeded the keystroke macro buffer size (currently 128
keystrokes).
\end{description}

\subsection[Compiler Error Messages]{Compiler Error Messages}

\begin{description}
\item[Ambiguous identifier]\mbox{}\\
You declared an identifier that has already been declared.

\item[Assignment incompatibility]\mbox{}\\
You are attempting an assignment of incompatible types.

\item[Bad comment]\mbox{}\\
You have not terminated a comment; the end of the file was reached
while still in the comment.

\item[Incompatible Types]\mbox{}\\
You are attempting to operate on types that are not compatible.

\item[Redefinition of Function's Parameters]\mbox{}\\
You have already declared the types of the function's parameters in
an earlier forward declaration.

\item[Invalid Integer]\mbox{}\\
You have used an integer constant which is out of the range --32768..+32767.

\item[Invalid Label Reference]\mbox{}\\
You have a GOTO to an undefined or invalid label.

\item[Invalid String]\mbox{}\\
A non-printable character was reached before the end of the string.
You may not have terminated the string properly.

\item[Invalid Assignment to Function Variable]\mbox{}\\
You cannot assign to the function variable at this scope level.

\item[Set Element Out of Range]\mbox{}\\
All set elements must have values in the range 0..63. This means,
for example, that sets of alphabetic characters are not allowed.

\item[Invalid Time Unit]\mbox{}\\
Valid timescales are HOURS, MINUTES, SECONDS, MILLISECONDS and
MICROSECONDS.

\item[Missing Queue Discipline - No Default]\mbox{}\\
You declared an interaction point without a queue discipline, and
no default queue discipline has been specified.

\item[Missing System Class]\mbox{}\\
You declared a module with body transitions but no process or
activity attribute.

\item[Too many labels]\mbox{}\\
You have exceeded the allowed number of labels.

\item[Too Many Levels]\mbox{}\\
You have exceeded the allowed number of scope levels.

\item[Invalid Ordinal Type]\mbox{}\\
An ordinal type was expected where some other type was used.

\item[Invalid Pointer Type]\mbox{}\\
A pointer type was expected where some other type was used.

\item[Redefinition of Procedure's Parameters]\mbox{}\\
You have already declared the types of the procedure's parameters in
an earlier forward declaration.

\item[Invalid Index Range]\mbox{}\\
A value which was out of the legal range was found. You may be
assigning an integer to a variable which is of some subrange or
enumeration type.

\item[Case Constant Repeated]\mbox{}\\
You have used a case constant more than once within the same case
statement.

\item[Role Repeated in Role List]\mbox{}\\
You have used the same role twice in a single role list.

\item[Invalid Type for Selector]\mbox{}\\
The case selector constant expression is of a different type to the
case selector index expression.

\item[Invalid Signed Constant]\mbox{}\\
Only integers may be signed; you have used a unary minus with some
other type.

\item[State Definition Repeated]\mbox{}\\
You have already had a state-definition part in the current set of
declarations; you may not have two.

\item[Invalid Type]\mbox{}\\
You have used a type in an inappropriate place.

\item[Undefined Identifier]\mbox{}\\
You have referenced an identifier that has not previously been
declared.

\item[Invalid Assignment to Control Variable]\mbox{}\\
You may not assign to loop control variables within loops.

\item[Invalid Non-local Control Variable]\mbox{}\\
Loop control variables must be local variables.

\item[Invalid Control Variable]\mbox{}\\
The variable you are using to control a loop is of an inappropriate
type.

\item[Label Reference Unresolved in Block]\mbox{}\\
You have a GOTO to a label which is never defined.

\item[Case index out of range]\mbox{}\\
You have used a case constant expression which is out of the
allowed range 0..255.

\item[Duplicate clause type in transition]\mbox{}\\
You have more than one clause of the same type associated
with the same transition block.

\item[ANY clause not implemented]\mbox{}\\
The ANY clause is not implemented in Version 1.2 of the {\em PEW}.

\item[ANY constants not implemented]\mbox{}\\
ANY constants must be of simple types. The PEW will compile a 
specification using ANY constants as thought the smallest
non-negative allowed value for the ANY constant was specified,
or if this is not possible, the largest negative value.

\item[... types not implemented]\mbox{}\\
The {\em PEW} only allows complete specifications that are
executable; ... types are thus not allowed.

\item[Dispose with tags not implemented]\mbox{}\\
Only the basic form of {\tt DISPOSE} is allowed; the variant form
is not implemented.

\item[EXTERNAL not implemented]\mbox{}\\
The {\em PEW} only allows complete specifications that are
executable; {\tt EXTERNAL} functions and procedures are thus not allowed.

\item[Functional parameters not implemented]\mbox{}\\
Functional parameters have not been implemented in Version 1.0 of the {\em PEW}.

\item[NEW with tags not implemented]\mbox{}\\
Only the basic form of {\tt NEW} is allowed; the variant form
is not implemented.

\item[Packed types not implemented]\mbox{}\\
The {\em PEW} does not support type packing.

\item[PRIMITIVE not implemented]\mbox{}\\
The {\em PEW} only allows complete specifications that are
executable; {\tt PRIMITIVE} functions and procedures are thus not allowed.

\item[Procedural parameters not implemented]\mbox{}\\
Procedural parameters have not been implemented in Version 1.0 of the {\em PEW}.

\item[Record type expected in WITH statement]\mbox{}\\
The identifier following {\tt WITH} is not a record.

\item[Estelle construct illegal in procedure/function]\mbox{}\\
You have used an Estelle construct that is prohibited within Pascal
functions and procedures.

\item[GOTO illegal in current context]\mbox{}\\
GOTOs are only allowed within procedures and functions.

\item[Illegal module variable reference]\mbox{}\\
You have used a module variable in a context where it may not be
used.

\item[Identifier expected]\mbox{}\\
The compiler expected an identifier but got something else.

\item[Integer expected]\mbox{}\\
The compiler expected an integer but got something else.

\item[TO or DOWNTO expected]\mbox{}\\
The keyword {\tt TO} or {\tt DOWNTO} was expected as the compiler
is processing a {\tt FOR} loop.

\item[Bad factor in expression]\mbox{}\\
You have an illegitimate identifier reference in an expression.

\item[Bad constant]\mbox{}\\
A constant was expected but not found.

\item[Bad formal parameter syntax]\mbox{}\\
A syntax error in a formal parameter list occurred.

\item[Clause or BEGIN expected]\mbox{}\\
The compiler expected a transition but got something else.

\item[Syntax error]\mbox{}\\
Some kind of syntax error occurred.

\item[Incompatible roles in CONNECT or ATTACH]\mbox{}\\
To {\tt ATTACH} two interaction points, they must have the same
roles. To {\tt CONNECT} two interaction points, they must have
opposite roles. You violated one of these rules.

\item[Syntax error: ';' expected]
\item[Syntax error: ':' expected]
\item[Syntax error: ')' expected]
\item[Syntax error: '(' expected]
\item[Syntax error: comma expected]
\item[Syntax error: END expected]
\item[Syntax error: DO expected]
\item[Syntax error: '[' expected]
\item[Syntax error: '.' expected]
\item[Syntax error: ']' expected]
\item[Syntax error: OF expected]\mbox{}\\
These are all self-explanatory errors, and are special cases of the
general `Syntax Error' error.

\item[Out of symbol table space]\mbox{}\\
The specification you are attempting to compile has exhausted the
symbol table space. Try to reduce the number of declarations you
have. Eliminate type declarations that are only used once, and add
type declaractions where these are implicit and used several times.

\item[String store overflow]\mbox{}\\
You have exhausted the string store for storing identifier names.
Try to shorten the names of identifiers to save space.

\item[Out of code space]\mbox{}\\
You have exhausted the size of the code buffer.

\item[Too many lines in source program]\mbox{}\\
Version 1.x of the PEW allows a maximum of 5000 source lines in
a single specification.

\item[Invalid type for exported variable]\mbox{}\\
You have declared an exported variable which is not a Pascal
variable type.

\item[Expected an exported variable reference]\mbox{}\\
An exported variable reference was found by the compiler, but the
variable referenced is not an exported variable.

\item[Expected module type or ordinal type in EXIST, FORONE or ALL]\mbox{}\\
An inappropriate identifier was used in one of these constructs.

\item[Only a single module variable allowed in EXIST, FORONE or ALL]\mbox{}\\
Module domains for these constructs may have only a single module
variable.

\item[Illegal mix of module and ordinal types in EXIST, FORONE or ALL]\mbox{}\\
These constructs allow both module and ordinal types, but not
simultaneously.

\item[Too many identifiers in domain list in EXIST, ALL or FORONE]\mbox{}\\
You have exceeded the maximum allowed length of a domain list.

\item[Illegal operation for pointer types]\mbox{}\\
You have attempted an operation on pointer types that is not
permitted.

\item[Illegal clause type in initialisation transition]\mbox{}\\
Initialisation transitions may only have {\tt TO} and {\tt
PROVIDED} clauses.

\item[Maximum number ({\em val}) of unique named transitions exceeded]\mbox{}\\
The PEW keeps track of the firing frequencies of all named transitions
for use in the ANALYSIS menu's dependent expression evaluator. As
space must be reserved for this data, there is a limit on how many named
transitions are permitted in a specification.

\item[Interaction {\em name} has no associated OUTPUT statement]\mbox{}\\
This interaction is used in WHEN clause(s) but does not occur in
any OUTPUT statements. The transition(s) that refer to it can thus
never be enabled, and are redundant.

\item[Interaction {\em name} has no associated WHEN clause]\mbox{}\\
This interaction is used in OUTPUT statement(s) but is never used
in any WHEN clauses. This is a potential deadlock situation due
to the strict FIFO nature of Estelle's queues.

\item[Interaction {\em name} is not referenced]\mbox{}\\
This interaction is never used in any WHEN clause or OUTPUT statement.

\item[State `{\em name}' is unreachable]\mbox{}\\
This state never occurs in a TO clause or a FROM..TO SAME clause pair.
Thus it is not possible for it to ever be reached. This is not an error,
but the state is redundant.

\item[State `{\em name}' is a deadlock state]\mbox{}\\
If this state is reached, it can never be left, as there are no FROM
clauses that are enabled in this state. The prescence of such a
state indicates a design error in your specification.

\item[Bad nesting of clauses]\mbox{}\\
You have badly nested clauses in the current transition. Specifically,
you have used a clause type that is already on the stack of clauses
for this transition.

\item[Line {\em n}: (E{\em m}) Interaction in OUTPUT may not have a priority]\mbox{}\\

This interaction is not associated with a priority-queueing channel
type.

\item[Line {\em n}: (E{\em m}) OUTPUT interaction cannot have a propogation delay]\mbox{}\\

This interaction is not associated with a FIFO-queueing channel type.

\end{description}


\subsection[Interpreter/Debugger Error Messages]{Interpreter/Debugger
Error Messages}

\begin{description}
\item[Cannot open log file]\mbox{}\\
The {\em PEW} failed to open the file {\tt ESTELLE.LOG}. You may
have some other application running, or your disk may be full. If you
are running from a floppy disk, you may have a write-protect sticker
on your disk.

\item[Expression value in CASE statement has no corresponding label]\mbox{}\\
You have not dealt explicitly with each possible case for the case
statement index expression type. 

\item[Attempt to dequeue nonexistent interaction]\mbox{}\\
This is an internal error. It should never occur.

\item[Attempt to output interaction to unconnected IP]\mbox{}\\
You have attempted to execute an {\tt OUTPUT} statement to an
unbound interaction point.

\item[Attempt to execute illegal instruction ({\em n}) at code offset {\em o}]\mbox{}\\
This is an internal error. It can only occur if stack overflow
occurred but was never detected. It is unlikely to ever happen.

\item[Stack overflow]\mbox{}\\
A process's stack has exceeded its bounds. Process stacks are
usually expanded dynamically as required, unless the process performs
dynamic memory allocation using {\tt NEW}, at which point the stack
is expanded by 1000 words and then frozen in size.

\item[Attempt to expand process stack failed]\mbox{}\\
As mentioned above, process stacks are expanded dynamically as
required. If this fails, it is due to either low memory or the stack
becoming too big ($\geq$32kB).

\item[Index or value {\em val} out of range {\em val}\dots{\em val}]\mbox{}\\
A variable of some ordinal type has exceeded its allowed range of
values.

\item[Attempt to connect an already connected interaction point]\mbox{}\\
An interaction point that is already bound was referenced in a {\tt
CONNECT} statement.

\item[Attempt to attach a bound interaction point]\mbox{}\\
An interaction point that is already bound was referenced in a {\tt
ATTACH} statement.

\item[Too many failed WHEN clauses in process]\mbox{}\\
As the {\em PEW} checks for interactions that cannot be dealt with,
it maintains data structures for this purpose. You are attempting to
execute a specification which exceeds the available data structure
space for this purpose.

\item[NEW failed to allocate heap memory]\mbox{}\\
An attempt to dynamically allocate memory failed because the
request was too large or the stack is close to overflowing.

\item[Attempt to DISPOSE illegal memory block]\mbox{}\\
An attempt to dispose a memory block that is not a currently valid
memory block allocated by {\tt NEW} was made. It is likely that you
have never done a {\tt NEW} on this pointer before calling {\tt
DISPOSE}.

\item[No breakpoint corresponding to this number]\mbox{}\\
A reference to a non-existent breakpoint was made.

\item[Division by zero]\mbox{}\\
The next divide instruction to be executed by the PEW has a 
zero-valued denominator.
\end{description}

\subsection[Memory Allocation Failure Messages]{Memory Allocation Failure
Messages}

\begin{description}
\item[Failed to allocate interaction]\mbox{}\\

\item[Failed to allocate a module entry]\mbox{}\\

\item[Failed to allocate a transition sequence table]\mbox{}\\

\item[Failed to allocate a module stack]\mbox{}\\

\item[Failed to allocate a module variable table]\mbox{}\\

\item[Failed to allocate an IP table]\mbox{}\\

\item[Failed to allocate a common queue]\mbox{}\\

\item[Failed to allocate a reception queue]\mbox{}\\

\item[Failed to allocate a transition table]\mbox{}\\

\item[Line {\em n}: (E{\em m}) Failed to allocate symbol table memory block]\mbox{}\\

\item[Line {\em n}: (E{\em m}) Failed to allocate compiler code buffer]\mbox{}\\

\item[Line {\em n}: (E{\em m}) Too many transitions in process]\mbox{}\\

The number of transitions in a process is limited to 64.
\end{description}

\newpage
\appendix
\section[The {\em PEW} Estelle Implementation]{The {\em PEW} Estelle
Implementation}

This section gives a short summary of how the
implementation of Estelle in the {\em PEW} differs from the Draft
International Standard.

The following restrictions apply to the {\em PEW}:

\begin{enumerate}
\item {\tt REAL} and {\tt PACKED} data types are not supported.
\item The {\tt ANY} clause is not supported.
\item The {\tt NEW} and {\tt DISPOSE} standard procedures do not
allow the variant forms.
\item Partial specifications are not supported; i.e., {\tt
PRIMITIVE} and {\tt EXTERNAL} functions and procedures and {\tt ANY} types.
\item Procedures and functions may not be passed as parameters.
\end{enumerate}
In addition, the compiler performs no purity checking of procedures
and functions. The keyword {\tt PURE} is ignored, and it is assumed
that the user takes responsibility for purity checking. 

The restriction in the standard of {\tt WHEN} and {\tt
DELAY} clauses being mutually exclusive is not enforced by the {\em
PEW} compiler. Instead, the combination is treated as follows: the
{\tt WHEN} clause must be continuously satisfied for the duration of 
the delay for the transition to be offered for execution. If at any stage 
during the delay the {\tt WHEN} clause is no longer satisfied, then the 
delay timer associated with the transition is reset.

Additions to the {\em PEW} include:

\begin{enumerate}
\item {\tt READ}, {\tt WRITE} and {\tt
WRITELN} procedures for integer, character and string variables and
constants. Fieldwidths are not supported. All I/O is directed to a
pop-up window which can be viewed at any time in the interpreter by
pressing {\em F7}. This can be overridden using a compiler directive
as follows:
\begin{itemize}
\item \{\$o0\} -- disable all output
\item \{\$o1\} -- send output to log file only
\item \{\$o2\} -- send output to window only (default)
\item \{\$o3\} -- send output to both log file {\em and} window
\end{itemize}
It should be noted that this compiler directive generates an e-code
instruction at the point where it occurs, so placement is important.
The advantage of generating an instruction is that different WRITE
statements can have different destinations. Once such an
instruction is processed, all WRITE statements will be subject to it
until the next such instruction is processed.

It should also be noted that when logging ({\em F8}) is turned on,
then the default is to send output to both the log file and the
window.
\item  A {\tt RANDOM} function has been added to generate random numbers.
This takes a single positive integer argument {\em n} and returns a
pseudo-random number in the range 0..{\em n}--1.
\item A {\tt GLOBALTIME} function has been added which returns the
current simulation time.
\item A {\tt FIRECOUNT} function has been added which returns the
number of times the containing transition has been executed.
\item A {\tt QLENGTH} function has been added which returns the
length of the reception queue of an interaction point. The
interaction point must be specified as an argument to the function.
\item {\tt OUTPUT} statements may be preceded by a compiler directive to 
specify	their reliability. For example {\tt \{\$R95\} OUTPUT...} specifies an
{\tt OUTPUT} statement that is 95\% reliable. This reliability applies
only for a single {\tt OUTPUT} statement, after which it is reset to
100\%.
\item The {\em PEW} detects interactions at the head of IP queues
that cannot be dealt with and discards these. Each time this occurs,
a message is written to the log file detailing the occurence. This
helps to check whether all possible interactions have been catered
for.
\item The {\em PEW} compiler detects unused and unreachable states,
interactions that can be sent but never received, or transitions that
are only enabled upon reception of interactions that are never sent.
Thus a reasonable degree of static validation is performed.
\item The {\em PEW} compiler allows the use of {\tt ANY} constants.
It uses the smallest possible positive value for such a constant, if
possible; otherwise the largest value is used.
Thus, ANY 10..20 will be treated as 10, ANY -10..10 will be treated
as zero, and ANY -20..-10 will be treated as -10. 
\item The {\em PEW} allows uniform, exponential, geometric and Poisson delay
distributions. Selecting between these is discussed in
Section~\ref{execenv}.
\item Normally, a {\tt WRITE} or {\tt WRITELN} statement directed at
the pop-up I/O window results in a short pause for you to examine the
result before execution proceeds. This can be disabled and enabled at
different points by using the {\em fast write} compiler directives
{\tt \{\$F--\}} and {\tt \{\$F+\}}.
\item The execution log contains transition sequence count and transition
sequence delay matrices for each process. In some cases, we may wish
to restrict attention to a subset of the transitions in a process.
This can be achieved to a limited extent by use of the {\tt \{\$T\}}
directive, which may be used once per module, and divides the
module's transitions up into two separate groups.
\item In Estelle, interactions are queued in FIFO order. This can be
overridden in the {\em PEW} by use of the {\tt \{\$QR\}} (random
queueing) and {\tt \{\$QP\}} (priority queueing) directives. These
directives affect the next channel type declaration, after which FIFO
queueing once again becomes the default. When declaring a channel
type, priorities are assigned sequentially to each set of
interactions declared, starting with zero (the highest priority). The
current priority value can be set to a specific value by using the
{\tt \{\$S{\em value}\}} directive. An example will help to
illustrate this:

\begin{verbatim}
{$QP Select priority queueing}

CHANNEL prot_prov_chan(prot_role, prov_role);
    BY prot_role, prov_role:
        {** REJ frames have highest priority **}
        REJ(sv:seq_type; rv:seq_type);
        {** RR frames have priority 1 **}
        RR(sv:seq_type; rv:seq_type);
        {** RNR has same priority as RR **}
        {$S1} RNR(sv:seq_type; rv:seq_type);
        { I frames have priority 2 which is the lowest }
        I(sv:seq_type; rv:seq_type; data:data_type;crc:crc_type);
\end{verbatim}

\item A propogation delay can be associated with an interaction point
by preceding the IP declaration with the directive {\tt \{\$D{\em
value}\}}. Interactions output through the IP will be timestamped,
and may only be dequeued when the global time exceeds their arrival
time. This facility is only supported for IPs that use FIFO queueing.
\end{enumerate}

\newpage
\section[An Overview of Estelle]{An Overview of Estelle}

\subsection[Introduction]{Introduction}
 
Estelle is a {\em formal description technique} (FDT) based on
an extended state transition model (Estelle is a loose
acronym for {\em Extended State Transition Language}). It is
being developed as an International Standards Organisation
standard by ISO/TC 97/SC 21, the subcommittee for
Information Retrieval, Transfer and Management for the Open
Systems Interconnection (OSI).
 
Estelle can be used to describe distributed, concurrent
information processing systems. It is aimed in particular
at describing the service definitions and protocol
specifications of OSI layers. In the OSI model, each layer
is separated from the layers above and below, and
communicates with these via service requests. Apart from
this communication, the various layers can be regarded as
separate asynchronous systems. Furthermore, the peer layers
on separate systems behave relatively asynchronously with
respect to one another. A further characteristic of such
systems is nondeterminism. Through a hierarchical structure
of processes, Estelle provides facilities for both
asynchronous parallel execution and nondeterministic
sequential execution.
 
Essentially, Estelle is a subset of the programming
language Pascal (ISO 7185) with extensions for concurrency
based on an extended finite-state model. An Estelle program
(called a {\bf specification}) specifies a number of processes
which may execute either sequentially or concurrently, or a
combination of both. Each process is an extended finite-
state automaton, consisting of a number of {\em states}, a set of
possible {\em state transitions}, the {\em conditions} which must be
satisfied for the transitions to be {\em enabled}, and the 
{\em actions} associated with each transition. The automata communicate
with one another using {\em message passing} on FIFO queues. Each
automaton typically represents a layer or part thereof of a
protocol executing on some specific machine. A 3-layer
protocol executing on two separate machines would thus
typically be represented in Estelle using 3*2=6 automata.
The entire system is dynamic and nondeterministic; new
processes or automata may be freely created and destroyed
during the execution of a specification.
 
 This informal introduction to the syntax
and semantics of Estelle assumes the reader is familiar
with the Pascal language; only the changes and extensions to
Pascal are considered.

\subsection[Conventions and Outline]{Conventions and Outline}
\label{conventions}
 
 Several conventions have been used throughout this
document. Within syntactic grammar rules, bold type has
been used to represent terminals, italic type for
nonterminals, and normal type for the grammar metalanguage.
The metalanguage consists of the symbols:

\begin{tabbing}
mmmmmm\=mmmmmmm\=\+\kill
=\>Separates right and left hand sides of productions\\
$|$\>Separates alternatives\\
$[]$\>Indicates optional elements\\
\{\}\>Indicates zero or more elements\\
+\{\}\>Indicates one or more elements\\
()\>Used for grouping purposes\\
$<>$\>Used to identify nonterminals\\
\dots\>Used to abbreviate a scalar range\\
\end{tabbing}

Unconventionally, productions are not terminated with a
period as this may cause confusion with punctuation symbols
that are intended as language terminals.
 
Within the productions, upper case nonterminals represent
Pascal constructs. In general, these productions are not
explicitly given, but should be fairly obvious to the reader
who is familiar with Pascal. Lower case nonterminals are
used exclusively for Estelle constructs.
 
Within the text, important terms have been indicated in
italic, while grammar terminals are once again indicated in
bold.
 
To clarify the purpose of identifiers, they are often
prefixed with an abbreviation to indicate their use. For
example, {\em role-IDENT} is an identifier whose purpose it is to
identify a role. These prefixes are included for semantic
clarification only; all identifiers ultimately have the same
production rule.

The various parts of a programming language are usually
strongly interconnected; it is difficult to discuss any one
aspect of the language in isolation from others. The
breakdown used here is necessarily artificial.
In an attempt to reduce forward references different parts
of the language will be introduced as they are used;
however, the detailed discussion of any particular part will
be confined to the relevant section.
 
The first section is a brief overview of the additions to
ISO Pascal. None of the additions are discussed in detail;
they are dealt with individually in the remaining sections.
Section \ref{restrict} considers the restrictions to ISO Pascal that
have been introduced.
 
Section \ref{terminology} introduces some important terminology, while
Section \ref{automata} is a discussion of how finite-state automata may
be realised using Estelle. It concentrates primarily on the
external behaviour of the automata, regarding each automaton
as a process in a multiprocess system. Section \ref{states}
discusses the internals of the automata, namely, the
specification of states and state transitions, and how these
are interpreted during execution. Section \ref{IPC} concentrates
on the communication between automata, that is, the message
passing and shared variable facilities of Estelle.
 
As each new aspect of the language is covered, the
syntactic production rules for that aspect are given, and
the semantics discussed. A collected syntax is included in
the appendix. Section \ref{conventions} describes the metalanguage used to
specify the syntactic productions, as well as other
conventions used.

\subsection[Additions to ISO Pascal]{Additions to ISO Pascal}

The main additions to standard Pascal are the provision
for concurrent systems and the provision for specifying
finite-state automata. Other additions include:

\begin{itemize}
\item identifier names may have underscores in any position;
this is the only difference from standard Pascal
identifiers. The general production rule for identifiers
is thus:

\begin{tabbing}
mmm\=mmmmmmm\=mm\=\+\kill
$<$IDENT$>$\>=\>$<$letter$>$ \{ $<$letter$>|<$digit$>$ \}\\
$<$letter$>$\>=\> a $|$ b $|$ c $|$\dots$|$ z $|$ A $|$ B $|$ C $|$\dots$|$ Z $|$ \_\\
$<$digit$>$\>=\>0 $|$ 1 $|$ 2 $|$\dots$|$ 9
\end{tabbing}

\item partial specifications are possible, allowing early
syntactic checking, separate compilation or the 
specification of protocols without specifying irrelevant details.
This is achieved by allowing constants to be declared with the 
value {\bf any} and variables to be declared with the type 
\dots. In addition to functions and procedures being able to be
forward-declared by qualifying them with the keyword {\bf forward},
the {\bf primitive} and {\bf external} keywords can be similarly
used to declare functions or procedures that are not defined 
within the particular file containing these definitions. An {\bf
external} routine must be declared within another file which is part
of the specification; a {\bf primitive} routine is one which is
provided by the language implementor (typically a library
function/procedure). 

This facility allows incomplete specifications to
be compiled and thus checked for syntactic errors,
access to external libraries of functions, as well as
the ability to omit irrelevant parts of the protocol
specification (for example, the type of a protocol
data unit is not needed to define the protocol
itself).

\item the {\bf all} repetitive statement has been added to the
existing {\bf repeat}, {\bf while} and {\bf for} statements.

\item the structured statement {\bf forone} has been added

\item the expression factor {\bf exist} has been added

\item the simple statements {\bf attach}, {\bf detach}, 
{\bf connect}, {\bf disconnect}, {\bf output}, {\bf init},
{\bf release} and {\bf terminate} have been added.
\end{itemize}

The {\bf init}, {\bf release}, {\bf terminate}, {\bf all} and
{\bf forone} statements and {\bf exist} expression are discussed 
in Section \ref{automata}; the {\bf attach}, {\bf detach},
{\bf connect}, {\bf disconnect}, and {\bf output} statements
are discussed in Section \ref{IPC}.

\subsection[Restrictions to ISO Pascal]{Restrictions to ISO
Pascal}
\label{restrict}

The fact that Estelle supports concurrency has
repercussions within the Pascal subset, primarily to allow
functions and procedures to be reentrant. Several
restrictions to procedures and functions have thus been
introduced. They may not reference non-Pascal objects,
although the Estelle statements {\bf all}, {\bf forone} and {\bf
exist} may
still be used on the usual Pascal ordinal types. Functions
must all be pure; that is, side-effects are prohibited.
This is enforced through several restrictions: parameters
may be passed by value only and may not contain any
pointers, references to global variables are prohibited and
so are calls to any other non-pure routines. Procedures
need not be pure; if they are, they may be declared as {\bf pure
procedure}, and may then be called from functions and other
pure procedures.
 
All file operations have been removed, including the file
type and predefined {\bf text} type. Thus the execution of an
Estelle specification is only meaningful in a controlled
environment, or if the implementor adds his or her own
{\bf primitive} library routines for input/output.
 
Labels and {\bf goto}s are also restricted - they may only
occur within procedures and functions, and labels may only
be associated with the end of the containing procedure or
function; in other words, a {\bf goto} acts similar to a {\bf
return} statement in a language such as 'C'.

\subsection[Terminology]{Terminology}
\label{terminology}

Several terms are used within Estelle to describe
processes and interprocess communication. Although these
terms are discussed within the relevant sections, it is
appropriate to introduce some here.
 
A {\em channel} is a bidirectional FIFO queue used for
interprocess communication. Each channel is identified by a
{\em channel identifier}. The traffic over a channel (which
consists of messages called {\em interactions}) can be different
for each of the two directions; each direction is associated
with a {\em channel role} which specifies what interactions may be
sent in that direction. The endpoints of a channel are
called {\em interaction points}.
 
An executing protocol {\em specification} in Estelle consists
of one or more {\em modules} in a hierachical structure. These
modules may have different {\em classes}: {\em activities} allow
nondeterministic sequential execution (ie, if the scheduler
must choose between a number of enabled activities, it will
choose only one, and the choice is nondeterministic), while
{\em processes} allow asynchronous parallel execution (ie, the
scheduler will execute all of the enabled processes). This
description is an oversimplification of what actually
occurs, but is sufficient for now.
 
Each module comprises two parts: the {\em module header}
specifies the external visibility of the module, while the
{\em module body} specifies its internal behaviour.
 
The conditions associated with transitions are called
{\em clauses}, and may be of several different types.

\subsection[Automata]{Automata}
\label{automata}

In this section, we will consider automata as processes
in a multiprocessing system. We are thus primarily
concerned with the creation and destruction of automata, not
with their internal behaviour or intercommunication, which
are discussed in sections \ref{states} and \ref{IPC} respectively.
 
At the top level, an Estelle program is a
$<${\em specification}$>$, identified by a name $<${\em spec-IDENT}$>$ and a
class $<${\em system-class}$>$, with an internal structure described
by a $<${\em body-defn}$>$. Queues for interprocess communication may
be {\em individual} or {\em shared}; usually such {\em queue
disciplines} will
be explicitly declared within queue declarations, although a
global default can be specified with the optional
$<${\em defaults}$>$. This default must be provided if there are any
queues declared without explicit queue disciplines.
 
Timescales are semantically meaningless and are simply
used to express the intentions of the implementor. Time in
the Estelle model need only be treated consistently; the
actual units are irrelevant (except in real
implementations), and serve only to express the intentions
of the implementor.

\begin{tabbing}
mm\=mmmmmmmm\=mm\=\+\kill
$<${\em specification}$>$\>=
\>{\bf specification} $<${\em spec-IDENT}$>$ [$<${\em system-class}$>$]{\bf
;}\\
\>\>[ $<${\em defaults}$>$ ]\\
\>\>[ $<${\em time-options}$>$ ]\\
\>\>$<${\em body-defn}$>$\\
\>\>{\bf end .}\\
\mbox{}\\
$<${\em system-class}$>$\>=\>{\bf systemprocess} $|$ {\bf systemactivity}\\
\mbox{}\\
$<${\em defaults}$>$\>=\>{\bf default} $<${\em q-discipline}$>$ {\bf ;}\\
\mbox{}\\
$<${\em q-discipline}$>$\>=\>{\bf common queue} $|$ {\bf individual queue}\\
\mbox{}\\
$<${\em time-options}$>$\>=\>{\bf timescale} $<${\em ts-IDENT}$>$ {\bf ;}\\
\mbox{}\\
$<${\em ts-IDENT}$>$\>=\>{\bf hours} $|$ {\bf minutes} $|$ {\bf seconds} $|$\\
\>\>{\bf milliseconds} $|$ {\bf microseconds}\\
\mbox{}\\
$<${\em body-defn}$>$\>=\>$<${\em decl-part}$>$\\
\>\>$<${\em init-part}$>$\\
\>\>$<${\em trans-decls}$>$\\
\end{tabbing}

The $<${\em body-defn}$>$ of a specification consists of three
parts: the {\em declarations}, an {\em initialisation part}, and a
{\em transition declaration part}. The initialisation and
transition declaration parts both specify groups of
transitions (see section \ref{states} for details). Initially it is
assumed that the specification exists in an initial state
that results from executing a transition from its $<${\em init-
part}$>$ if this is non-empty; otherwise it is in a {\em pre-initial
state}.

\begin{tabbing}
mm\=mmmmmmmm\=mm\=\+\kill
$<${\em decl-part}$>$\>=\>{ $<${\em decls}$>$ }\\
\mbox{}\\
$<${\em decls}$>$\>=\>$<${\em CONSTANT-DEFN-PART}$>$\\
\>\>$|<${\em TYPE-DEFN-PART}$>$\\
\>\>$|<${\em channel-defn}$>$\\
\>\>$|<${\em module-header-defn}$>$\\
\>\>$|<${\em module-body-defn}$>$\\
\>\>$|<${\em ip-decls}$>$\\
\>\>$|<${\em mod-var-decls}$>$\\
\>\>$|<${\em VARIABLE-DECL-PART}$>$\\
\>\>$|<${\em state-defns}$>$\\
\>\>$|<${\em state-set-defns}$>$\\
\>\>$|<${\em PROCEDURE-AND-FUNCTION-DECL-PART}$>$\\
\mbox{}\\
$<${\em init-part}$>$\>=\>\{ {\bf initialise} $<${\em trans-group}$>$ \}\\
\mbox{}\\
$<${\em trans-decls}$>$\>=\>\{ $<${\em trans-decl}$>$ \}\\
\mbox{}\\
$<${\em trans-decl}$>$\>=\>{\bf trans} $<${\em trans-group}$>$\\
\mbox{}\\
$<${\em trans-group}$>$\>=\>+\{ $<${\em clause-group}$><${\em trans-block}$>$
{\bf ;} \}\\
\end{tabbing}

The declarations may be in any order, and, with the
exception of the $<${\em state-defn-part}$>$, any declaration or
definition part may occur more than once. Apart from the
usual Pascal objects, the declarations may contain {\em channels}
(bidirectional FIFO message queues) and {\em interaction points}
(queue endpoints), module {\em headers}, {\em bodies} and {\em
variables}, and {\em states} and {\em state sets}.

An executing Estelle specification consists of
a tree of {\em instances} of generic {\em modules}. The
process structure may be dynamic; processes may at any time
instantiate others with {\bf init} statements and kill children
processes with the {\bf release} or {\bf terminate} statements. These
are discussed later in this section.

An Estelle module is described by its external visibility
(which is defined by the $<${\em modheader-defn}$>$) and internal 
behaviour (which is defined by the $<${\em modbody-defn}$>$).
These two components are separate, and thus modules with the
same external visibility may exhibit different internal
behaviour. Modules are bound to headers at compile time and bodies at
instantiation time.

\begin{tabbing}
mm\=mmmmmmmmmm\=mm\=\+\kill
$<${\em modheader-defn}$>$\>=
\>{\bf module} $<${\em modhdr-IDENT}$>$ [ $<${\em class}$>$ ]\\
\>\>[ {\bf (} $<${\em param-list}$>$ {\bf )} ] {\bf ;}\\
\>\>[ {\bf ip} +\{ $<${\em ip-decl}$>$ {\bf ;} \} ]\\
\>\>[ {\bf export} +\{ $<${\em VARIABLE-DECL}$>$ {\bf ;} \} ]\\
\>\>{\bf end ;}\\
\mbox{}\\
$<${\em class}$>$\>=\>{\bf systemprocess} $|$ {\bf systemactivity}\\
\>\>$|$ {\bf process} $|$ {\bf activity}\\
\mbox{}\\
$<${\em param-list}$>$\>=\>$<${\em VALUE-PARAM-SPECIFICATION}$>$\\
\>\>\{ {\bf ;} $<${\em VALUE-PARAM-SPECIFICATION}$>$ \}
\end{tabbing}

The only external access to a module instance is via its
interaction points and {\em exported variables} which are shared
with the instance's parent only. These exported variables
are declared in the $<${\em modheader-defn}$>$. The module header
also defines parameters which are passed (by value only) to
an instance of the module when the instance is created. The
interaction points declared in the module header are
{\em external interaction points}; in other words, they may be
accessed by the module's parent.

\begin{tabbing}
mm\=mmmmmmmm\=mm\=\+\kill
$<${\em modbody-defn}$>$=\>\>{\bf body} $<${\em modbod-IDENT}$>$ {\bf for} $<${\em
modhdr-IDENT}$>$ {\bf ;}\\
\>\>( $<${\em body-defn}$>$ {\bf end ;}\\
\>\>$|$ {\bf external ;} )
\end{tabbing}

Notice that two identifiers are used to identify the
module body; the first is associated with the module body
itself, and the second specifies which module header is
being referenced. If the body definition contains the
keyword {\bf external}, then the actual body definition is in some
other specification.

The relative behaviour of module instances in the
hierarchy is determined by the nesting of their definitions
as well as the class qualifiers {\bf systemprocess},
{\bf systemactivity}, {\bf process} and {\bf activity}. Processes allow for
synchronous parallel execution, while activities allow
nondeterministic sequential execution. A process may be
created and released only by its parent. A module with a
class qualifier is called {\em attributed}; the only non-
attributed modules are inactive modules, which as their name
suggests consist only of initialisation parts. Inactive
modules may only parent system modules, which may in turn
parent only process or activity modules. (System)process
modules may contain either process or activity children,
while (system)activity modules may contain only other
activity modules.

A consequence of the above is that any initial state of a
specification defines a fixed number of system instances and
interaction point links above them; this structure once
initialised cannot change since the parent modules of the
systems are all inactive and hence cannot create any new
processes after initialisation. Furthermore, as their
parents are inactive and hence have no execution priority
over the systems, the systems behave fully asynchronously
with respect to one another.

Instances of modules are identified by {\em module variables}.
References to exported variables and interaction points must
be qualified by the name of the appropriate module variable.
Arrays of module variables are allowed.

\begin{tabbing}
mm\=mmmmmmmm\=mm\=mmmm\=\+\kill
$<${\em mod-var-decls}$>$=\>\>{\bf modvar} +\{ $<${\em mod-var-decl}$>$ {\bf
;} \}\\
\mbox{}\\
$<${\em mod-var-decl}$>$\>=\>$<${\em IDENT-LIST}$>$ {\bf :} $<${\em
modhdr-IDENT}$>$\\
\>\>$|<${\em IDENT-LIST}$>$ {\bf : array [} $<${\em index-type-list}$>$ {\bf
]}\\
\>\>\>{\bf of} $<${\em modhdr-IDENT}$>$\\
\end{tabbing}

The initial value of a module variable is undefined; an
{\bf init} statement will initialise the variable, and a {\bf
release} or {\bf terminate} statement causes it to once 
again become undefined.

\begin{tabbing}
mm\=mmmmmmmmmmmm\=mm\=mmm\=\+\kill
$<${\em init-stmnt}$>$\>=\>{\bf init} $<${\em module-var}$>$ {\bf with} $<${\em
modbod-ident}$>$\\
\>\>\>[ {\bf (} $<${\em EXPR-LIST}$>$ {\bf )} ]\\
\mbox{}\\
$<${\em release-stmnt}$>$\>=\>{\bf release} $<${\em module-var}$>$\\
\mbox{}\\
$<${\em terminate-stmnt}$>$\>=\>{\bf terminate} $<${\em module-var}$>$
\end{tabbing}

The execution of the {\bf init} statement causes a new instance
of the specified module body (and thus header) to be
created, its parameters to be initialised with the
expression values, and an initialisation transition from the
module body (if one exists and is enabled) to be executed.
After execution of the {\bf init} statement, the module variable
identifies the module instance.
 
The {\bf release} statement allows a module to release children
module instances. All external interaction points of the
module instance are detached or disconnected, and the module
instance and {\em all its descendants} are released and are no
longer available. The {\bf terminate} statement is a similar
statement, except that it uses a form of detach known as
{\em simple detach}. This will be described later.
 
The {\bf all} statement is a repetitive statement which allows
iteration over a normal ordinal type or over a set of module
instances.

\begin{tabbing}
mm\=mmmmmmmmmmmm\=mm\=mmm\=\+\kill
$<${\em all-stmnt}$>$\>=\>{\bf all} ($<${\em domain-list}$>|<${\em
module-domain}$>$)\\
\>\>\>{\bf do} $<${\em STATEMENT}$>$\\
\mbox{}\\
$<${\em domain-list}$>$\>=\>$<${\em IDENT-LIST}$>$ {\bf :} $<${\em
ORDINAL-TYPE}$>$\\
\>\>\>\{ {\bf ;} $<${\em IDENT-LIST}$>$ {\bf :} $<${\em ORDINAL-TYPE}$>$
\}\\
\mbox{}\\
$<${\em module-domain}$>$\>=\>$<${\em IDENT}$>$ {\bf :} $<${\em modhdr-IDENT}$>$
\end{tabbing}

The result of an {\bf all} statement is the execution of the
$<${\em STATEMENT}$>$ for either:

\begin{itemize}
\item all vectors of values of ordinal type(s) given in the
 $<${\em domain-list}$>$; or

\item all children module instances whose header definitions are
 identified by the $<${\em modhdr-IDENT}$>$ in the $<${\em module-
 domain}$>$.
\end{itemize}

The order of execution is arbitrary. If the domain is
empty, the statement is not executed. The bounds of a
domain are evaluated once and are not affected by execution
of the statement.
 
A similar statement is the {\bf forone} statement:

\begin{tabbing}
mm\=mmmmmmmmm\=mm\=mmmm\=\+\kill
$<${\em forone-stmnt}$>$\>=\>
{\bf forone} ($<${\em domain-list}$>|<${\em module-domain}$>$)\\
\>\>\>{\bf suchthat} $<${\em BOOLEAN-EXPR}$>$\\
\>\>\>{\bf do} $<${\em STATEMENT}$>$\\
\>\>\>[ {\bf otherwise} $<${\em STATEMENT}$>$ ]
\end{tabbing}

This is much like the {\bf all} statement, except that only one
of the possible (vector of) values is used, and this must
satisfy the Boolean expression.
 
To assist in these statements, the {\bf exist} Boolean
expression may be used:

\begin{tabbing}
mm\=mmmmmmmm\=mm\=mmmm\=\+\kill
$<${\em exist-one}$>$\>=\>{\bf exist} ($<${\em domain-list}$>|<${\em
module-domain}$>$)\\
\>\>\>{\bf suchthat} $<${\em FACTOR}$>$\\
\end{tabbing}

The semantics are obvious in the light of the {\bf forone}
statement: if an element of the domain exists that satisfies
the $<${\em FACTOR}$>$, then the expression returns TRUE, otherwise it
returns FALSE.

\subsection[States and State Transitions]{States and State
Transitions}
\label{states}

We now consider the details of how Estelle allows state
transitions to be specified.
 
Each instance of a module is a nondeterministic state
transition system, characterised by a set of states, subset
of initial states, and next-state relation. The initial
states are defined by the module $<${\em init-part}$>$, which
describes the initial state, variable initialisation, and
the initial hierarchy and interconnection structure of
descendant module instances, if any. The one-to-many next-state
relation is specified by transitions, each composed of
a $<${\em clause-group}$>$ determining the pre- and post- states and
any conditions which must be fulfilled for the transition to
be {\em enabled}, and a $<${\em trans-block}$>$ defining the actions to be
executed upon firing the transition. If the transition
block is empty, the module is {\em inactive} and serves only as a
parent to other modules.

\begin{tabbing}
mm\=mmmmmmmm\=mm\=mmmm\=\+\kill
$<${\em init-part}$>$\>=\>\{ {\bf initialise} $<${\em trans-group}$>$ \}\\
\mbox{}\\
$<${\em trans-group}$>$\>=\>+\{ $<${\em clause-group}$><${\em trans-block}$>$
{\bf ;} \}\\
\end{tabbing}

All possible values of the control state of the EFSM must
be enumerated in a state definition part:

\begin{tabbing}
mm\=mmmmmmmm\=mm\=mmmm\=\+\kill
$<${\em state-defns}$>$\>=\>{\bf state} $<${\em IDENT-LIST}$>$ {\bf ;}
\end{tabbing}

States may be grouped together into state sets; this is
simply a compact notation allowing a single {\em state set
identifier} to reference a list of elements.

\begin{tabbing}
mm\=mmmmmmmmmmmmm\=mm\=mmmm\=\+\kill
$<${\em state-set-defns}$>$\>=\>{\bf stateset} +\{ $<${\em state-set-defn}$>$ {\bf ;}
\}\\
\mbox{}\\
$<${\em state-set-defn}$>$\>=\>
$<${\em state-set-IDENT}$>$ {\bf =} $<${\em state-set-constant}$>$\\
\mbox{}\\
$<${\em state-set-constant}$>$\>=\>[ $<${\em state-IDENT}$>$ \{ {\bf ,}
 $<${\em state-IDENT}$>$ \} ]\\
\end{tabbing}

Transitions consist of $<${\em clause-group}$>$s and 
$<${\em trans-block}$>$s:

\begin{tabbing}
mm\=mmmmmmmmmmmmm\=mm\=mmmm\=\+\kill
$<${\em trans-decl-part}$>$\>=\>\{ $<${\em trans-decl}$>$ \}\\
\mbox{}\\
$<${\em trans-decl}$>$\>=\>{\bf trans} $<${\em trans-group}$>$\\
\mbox{}\\
$<${\em trans-group}$>$\>=\>+\{ $<${\em clause-group}$><${\em trans-block}$>$ {\bf ;}
\}\\
\mbox{}\\
$<${\em clause-group}$>$\>=\>[ $<${\em provided-clause}$>$ ]\\
\>\>[ $<${\em from-clause}$>$ ]\\
\>\>[ $<${\em to-clause}$>$ ]\\
\>\>[ $<${\em any-clause}$>$ ]\\
\>\>[ $<${\em delay-clause}$>$ ]\\
\>\>[ $<${\em when-clause}$>$ ]\\
\>\>[ $<${\em priority-clause}$>$ ]\\
\mbox{}\\
$<${\em trans-block}$>$\>=\>$<${\em CONSTANT-DEFN-PART}$>$\\
\>\>$<${\em TYPE-DEFN-PART}$>$\\
\>\>$<${\em VAR-DECL-PART}$>$\\
\>\>$<${\em PROC-AND-FUNC-DECL-PART}$>$\\
\>\>[ $<${\em trans-name}$>$ ] $<${\em STATEMENT-PART}$>$\\
\mbox{}\\
$<${\em trans-name}$>$\>=\>{\bf name} $<${\em IDENT}$>$ {\bf :}\\
\end{tabbing}

The ordering of the clauses is not important.
Transitions with {\bf when} clauses are called {\em input
transitions}
as they are enabled only when some input condition occurs
(i.e. some message is received); those without are {\em spontaneous
transitions}. {\bf when} and {\bf delay} clauses are mutually exclusive.
The optional $<${\em trans-name}$>$ is for documentation purposes
only, and has no semantic effect.

Notice that the $<${\em trans-decl-part}$>$ consists of zero or more
$<${\em trans-decl}$>$s, each of which consists of a number of
transitions preceded by the keyword {\bf trans}. This grouping is
important, and affects two aspects: the nesting of clauses, and the
interpretation of {\bf provided otherwise} clauses. We shall discuss
the latter below. The nesting of clauses allows us to abbreviate
transitions which have clauses in common. Clauses are held on a
stack, which is affected in the following ways:

\begin{itemize}
\item upon entering a new $<${\em trans-decl}$>$, the stack is cleared.
\item upon entering a new $<${\em clause-group}$>$, the stack is popped
up to and including the clause type corresponding to the first clause
in the $<${\em clause-group}$>$. For example, if the new $<${\em
clause-group$>$} begins with a {\bf provided} clause, the stack will be
popped up to and including the {\bf provided} clause on the stack, if
any. If no such clause exists, the entire stack is popped.
\item subsequent to this, all clauses in the new $<${\em clause-group}$>$
are pushed on to the stack, one by one.
\item Finally, when the $<${\em trans-block}$>$ is entered, all clauses
currently on the stack are applicable to that block.
\item Once clauses are being pushed on to the stack, no duplicates
are allowed. For example, if a {\bf provided} clause is already
stacked, the occurence of a {\bf provided} clause is illegal.
\end{itemize}

We will now examine each type of clause in more detail.

\begin{tabbing}
mm\=mmmmmmmmmmmm\=mm\=mmmm\=\+\kill
$<${\em provided-clause}$>$\>=\>{\bf provided} ($<${\em BOOLEAN-EXPR}$>|$ {\bf otherwise})
\end{tabbing}

The {\bf provided} clause specifies a Boolean condition which
must be met for the transition to be enabled. If it is
omitted, it is assumed TRUE. {\bf otherwise} is only permitted as
the last clause of a transition group; it is TRUE is all other
{\bf provided} clauses in the transition group are FALSE.

\begin{tabbing}
mm\=mmmmmmmmmmmm\=mm\=mmmm\=\+\kill
$<${\em from-clause}$>$\>=\>{\bf from} $<${\em from-list}$>$\\
\mbox{}\\
$<${\em from-list}$>$\>=\>$<${\em from-element}$>$ \{ {\bf ,} $<${\em from-element}$>$
\}\\
\mbox{}\\
$<${\em from-element}$>$\>=\>$<${\em state-IDENT}$>|<${\em
state-set-IDENT}$>$\\
\mbox{}\\
$<${\em to-clause}$>$\>=\>{\bf to} $<${\em to-element}$>$\\
\mbox{}\\
$<${\em to-element}$>$\>=\>{\bf same} $|<${\em state-IDENT}$>$
\end{tabbing}

The $<${\em from-list}$>$ in a $<${\em from-clause}$>$ specifies those states
from which a transition may be validly executed; if it is
omitted, it is assumed satisfied. The $<${\em to-element}$>$
specifies the next state after the execution of the
transition; if it is omitted or same then the state does not
change.

\begin{tabbing}
mm\=mmmmmmmmmmmm\=mm\=mmmm\=\+\kill
$<${\em when-clause}$>$\>=\>{\bf when} $<${\em when-ip-ref}$>$ {\bf .} $<${\em
interaction-IDENT}$>$\\
\>\>\>[ $<${\em interaction-arg-list}$>$ ]\\
\mbox{}\\
$<${\em when-ip-ref}$>$\>=\>$<${\em ip-IDENT}$>$ [ {\bf [} $<${\em ip-index}$>$
\{ {\bf ,} $<${\em ip-index}$>$ \} {\bf ]} ]\\
\mbox{}\\
$<${\em ip-index}$>$\>=\>$<${\em CONSTANT}$>|<${\em VAR-IDENT}$>$\\
\mbox{}\\
$<${\em interaction-arg-list}$>$\>=\>( $<${\em interaction-arg-IDENT}$>$\\
\>\>\>\{ {\bf ,} $<${\em interaction-arg-IDENT}$>$ \} )\\
\end{tabbing}

A {\bf when} clause is satisfied if the interaction specified
by the $<${\em interaction-IDENT}$>$ is at the head of the queue
associated with the interaction point indicated by the
$<${\em when-ip-ref}$>$. The interaction at the head of the queue is
only dequeued as part of the execution of the transition.
The $<${\em interaction-arg-list}$>$ is optional; if it is included,
it must be complete and in the same order as in the
corresponding $<${\em interaction-defn}$>$. Section \ref{IPC} discusses
interactions in more detail.

\begin{tabbing}
mm\=mmmmmmmmmmmm\=mm\=mmmm\=\+\kill
$<${\em delay-clause}$>$\>=\>{\bf delay (} ($<${\em EXPR}$>$ {\bf ,} $<${\em
EXPR}$>$\\
\>\>\>$|<${\em EXPR}$>$ {\bf , *}\\
\>\>\>$|<${\em EXPR}$>$)\\
\>\>\>{\bf )}
\end{tabbing}

Some transitions may contain {\bf delay} clauses specifying a
minimum and maximum delay time; provided these are
interpreted consistently time in the Estelle model is
considered implementation dependent. A {\bf delay} clause
 
\begin{tabbing}
mm\=mmmmmmmm\=mm\=mmmm\=\+\kill
\>{\bf delay (E1,E2)}\\
\end{tabbing}
 
specifies that the transition should not be selected for
execution before E1 consecutive time-units of it being
enabled have elapsed. Once E1 has elapsed, the transition
may or may not be selected for execution (due to
nondeterminism), even if it is the only transition enabled.
However, once E2 units have elapsed, if the transition is
the only one enabled it must be selected. ``{\bf delay (E)}'' is
the same as ``{\bf delay (E,E)}''. An asterisk for the second
expression indicates that there is no upper bound, so it is
possible the transition may never be selected for execution.

\begin{tabbing}
mm\=mmmmmmmmmmm\=mm\=mmmm\=\+\kill
$<${\em priority-clause}$>$\>=\>{\bf priority} $<${\em priority-constant}$>$\\
\mbox{}\\
$<${\em priority-constant}$>$\>=\>$<${\em UNSIGNED-INTEGER}$>|<${\em
CONSTANT-IDENT}$>$\\
\end{tabbing}

{\bf priority} clauses are used to attach priorities to
transitions. Note that priorities are only relevant amongst
peer processes, as parent transitions always have priority
over children transitions. If the {\bf priority} clause is
omitted, the lowest priority is assumed. Zero is the
highest priority, after which they decrease systematically.

\begin{tabbing}
mm\=mmmmmmmm\=mm\=mmmm\=\+\kill
$<${\em any-clause}$>$\>=\>{\bf any} $<${\em domain-list}$>$ {\bf do}\\
\mbox{}\\
$<${\em domain-list}$>$\>=\>$<${\em IDENT-LIST}$>$ {\bf :} $<${\em
ORDINAL-TYPE}$>$\\
\>\>\>\{ {\bf ;} $<${\em IDENT-LIST}$>$ {\bf :} $<${\em ORDINAL-TYPE}$>$
\}\\
\end{tabbing}

{\bf any} clauses provide a shorthand way of specifying a
number of clauses. A transition with an any clause is
equivalent to as many transitions as there are distinct
values in the $<${\em ORDINAL-TYPE}$>$, one for each possible value.
The identifiers in an {\bf any} clause will typically be array
indexes used to refer to arrays of interaction points in
{\bf provided} clauses.
 
The EFSM represented by a module body has all its initial
states defined in the initialisation part:

\begin{tabbing}
mm\=mmmmmmmm\=mm\=mmmm\=\+\kill
$<${\em init-part}$>$\>=\>\{ {\bf initialize} $<${\em trans-group}$>$ \}\\
\end{tabbing}

The initialisation transition group is executed only once
when a module instance is created. Only one transition
block is executed, even if several are enabled (the choice
being nondeterministic). Only {\bf to} and {\bf provided} clauses are
allowed in the initialisation part.

As was mentioned in the previous section, systems allow
for synchronous parallel execution, while activities allow
nondeterministic sequential execution. The fact that we
have both concurrency and nondeterminism means that at any
stage, several transitions may be offered for execution. A
transition offered by a parent always has priority over one
offered by a child. If a parent offers no transition but
its children do, we first determine, for each child, the
transition(s) selected for execution. If the parent is a
(system)process then all of these transitions are selected
for execution; otherwise one child is selected (the choice
is nondeterministic) and its transition(s) used. This may
be made more precise as follows:

Consider the process hierarchy as a tree. The
transitions selected for execution ({\em $AS(gid_{P})$}) may be
synthesised bottom up using the following definition for any
particular non-leaf node {\em $P$} with children {\em $P_{i}$}:
 
\begin{itemize}
\item If {\em $P$} is active and offers a non-null transition, then this
 transition is selected (Parent priority).
\item Otherwise, if {\em $P$} is an (system)activity, then {\em
$AS(gid_{P})$} is one of the {\em $AS(gid_{P_{i}})$} (the choice is
nondeterministic).
\item Otherwise, if {\em $P$} is a (system)process, then {\em
$AS(gid_{P})$} is the union of all the {\em $AS(gid_{P_{i}})$} (parallel execution).
\end{itemize}
 
The execution of a specification thus proceeds as
follows: for each system, a set of transitions are selected
for execution. Once selected, a transition must eventually
execute. Once all of the selected transitions have been
executed for all systems, a new set is selected for each
system, and the process repeats.
 
State transitions are atomic; thus, for the execution of
Estelle specifications on uniprocessing systems, a process
scheduler may be invoked upon the completion of any state
transition to select the next transition (and hence process)
for execution.


\subsection[Interprocess Communication]{Interprocess
Communication}
\label{IPC}

Interprocess communication in Estelle is by message or
{\em interaction} passing on bidirectional FIFO message queues
({\em channels}) connected at their endpoints ({\em interaction
points}). As the process hierarchy may be an arbitrary
structure, the interconnections between processes may span
several levels of the hierarchy. Estelle does not allow the
direct connection of any one interaction point to any other
interaction point; instead, a connection between two
processes must be established by a common parent of both
processes (which may in fact be one of the processes
involved). The Estelle {\bf attach} statement then allows for the
extension of the connection down the process hierarchy as
necessary. It is important to distinguish between {\bf attach}
and {\bf connect}; {\bf connect} is used to establish connections while
{\bf attach} is used to extend those connections down through the
process hierarchy.

Each interaction point may be used to send interactions,
and is also associated with a FIFO queue (which may be
shared by other interaction points of the same module) from
which it can receive interactions. The set of interactions
which may validly be sent and received through an
interaction point are determined by the interaction point's
$<${\em channel-defn}$>$.

\begin{tabbing}
mm\=mmmmmmmmmmn\=mm\=mm\=\+\kill
$<${\em channel-defn}$>$\>=\>$<${\em chan-heading}$><${\em chan-block}$>$\\
\mbox{}\\
$<${\em chan-heading}$>$\>=\>{\bf channel} $<${\em chan-IDENT}$>$ {\bf (} $<${\em role-list}$>$
{\bf ) ;}\\
\mbox{}\\
$<${\em role-list}$>$\>=\>$<${\em role-IDENT}$>$ {\bf ,} $<${\em role-IDENT}$>$\\
\mbox{}\\
$<${\em chan-block}$>$\>=\>+\{ $<${\em interaction-group}$>$ \}\\
\mbox{}\\
$<${\em interaction-group}$>$\>=\>{\bf by} $<${\em role-IDENT}$>$ [ {\bf ,} $<${\em role-IDENT}$>$ ] {\bf
:}\\
\>\>\>+\{ $<${\em interaction-defn}$>$ \}\\
\mbox{}\\
$<${\em interaction-defn}$>$\>=\>$<${\em interaction-IDENT}$>$\\
\>\>\>[ {\bf (} $<${\em VALUE-PARAM-SPECIFICATION}$>$\\
\>\>\>\{ {\bf ;} $<${\em VALUE-PARAM-SPECIFICATION}$>$ \} {\bf )} ] {\bf ;}\\
\end{tabbing}

As can be seen, an interaction can have a number of value
parameters; none of these may be of a pointer-containing
type. For a given interaction point, a module assumes a
role declared by a $<${\em role-IDENT}$>$. A module may send
interactions associated with its assumed role; a module that
assumes the opposite role may receive interactions
associated with the first role.
 
An interaction point is an abstract bidirectional
interface through which a module may send or receive
interactions. Each interaction point has three attributes:

\begin{itemize}
\item the channel
\item the role
\item the queueing discipline
\end{itemize}

Any interaction associated with the role may be sent
through the interaction point; any interaction associated
with the opposite role may be received through the
interaction point.
 
The queueing discipline determines whether the queue
associated with that interaction point may be shared with
other interaction points of that module instance or not
(common or individual).

\begin{tabbing}
mm\=mmmmmmmmmmm\=mm\=mmmm\=\+\kill
$<${\em ip-decls}$>$\>=\>{\bf ip} +\{ $<${\em ip-decl}$>$ {\bf ;} \}\\
\mbox{}\\
$<${\em ip-decl}$>$\>=\>$<${\em IDENT-LIST}$>$ {\bf :} $<${\em ip-type}$>$\\
\>\>$|<${\em IDENT-LIST}$>$ {\bf : array [} $<${\em index-type-list}$>$ {\bf
]}\\
\>\>\>{\bf of} $<${\em ip-type}$>$\\
\mbox{}\\
$<${\em ip-type}$>$\>=\>$<${\em chan-IDENT}$>$ {\bf (}$<${\em role-IDENT}$>${\bf )}
[$<${\em q-discipline}$>$]\\
\mbox{}\\
$<${\em index-type-list}$>$\>=\>$<${\em INDEX-TYPE}$>$ \{ {\bf ,} $<${\em INDEX-TYPE}$>$
\}\\
\end{tabbing}

Interaction points may be referenced only by the
operations {\bf connect}, {\bf attach}, {\bf disconnect}, {\bf
detach}, the {\bf when}
clause of a transition, and in the {\bf output} statement.
Internal interaction points, which are defined in the $<${\em decl-
part}$>$ of a $<${\em body-defn}$>$, may only be bound and unbound using
{\bf connect} and {\bf disconnect}.

\begin{tabbing}
mm\=mmmmmmmmmmmm\=mm\=mmmm\=\+\kill
$<${\em connect-stmnt}$>$\>=\>{\bf connect} $<${\em connect-ip-ref}$>$ {\bf to} $<${\em
connect-ip-ref}$>$\\
\mbox{}\\
$<${\em connect-ip-ref}$>$\>=\>$<${\em internal-ip-ref}$>|<${\em
child-extern-ip-ref}$>$\\
\mbox{}\\
$<${\em disconnect-stmt}$>$\>=\>{\bf disconnect} ($<${\em connect-ip-ref}$>|<${\em
module-var}$>$)\\
\mbox{}\\
$<${\em attach-stmnt}$>$\>=\>{\bf attach} $<${\em external-ip-ref}$>$
{\bf to} $<${\em child-extern-ip-ref}$>$\\
\mbox{}\\
$<${\em detach-stmnt}$>$\>=\>{\bf detach} ($<${\em external-ip-ref}$>$
$|<${\em child-extern-ip-ref}$>$)\\
\mbox{}\\
$<${\em internal-ip-ref}$>$\>=\>$<${\em ip-ref}$>$\\
\mbox{}\\
$<${\em external-ip-ref}$>$\>=\>$<${\em ip-ref}$>$\\
\mbox{}\\
$<${\em child-extern-ip-ref}$>$\>=\>$<${\em module-variable}$>${\bf .}$<${\em
external-ip-ref}$>$\\
\mbox{}\\
$<${\em ip-ref}$>$\>=\>$<${\em ip-IDENT}$>$ [ {\bf [} $<${\em
INDEX-EXPR}$>$\\
\>\>\>\{ {\bf ,} $<${\em INDEX-EXPR}$>$ \} {\bf ]} ]\\
\end{tabbing}

The two interaction points referenced must be declared
with identical channel identifiers, and opposite role
identifiers for {\bf connect} or identical role identifiers for
attach. After execution of the {\bf connect} statement, any
interactions output through an interaction point are
received at the other (or its descendants if that
interaction point is attached). A {\bf connect} statement may
connect internal interaction points and child external
interaction points together in any combination.
 
The {\bf disconnect} statement causes the specified interaction
point, or all connected external interaction points of the
specified child module instance to be unbound. The
interactions for the interaction point(s) still remain in
the queue, and may still be processed.
 
The {\bf attach} statement attaches two interaction points, the
first of which may be currently bound and the second of
which may later be bound by an action of the child. An
interaction point at the end of a sequence of bound
interaction points is a {\em connection endpoint}. {\bf output}
statements at connection endpoints always cause their
interactions to be queued at the interaction point bound to
the opposite connection endpoint.
 
When the {\bf attach} statement is executed, interactions
present in the external interaction point queue which came
through this queue are removed, and appended to the queue of
the external interaction point of the lowest level
descendant module instance which attached to the child
external interaction point. The interactions for two
interaction points whose queuing option is individual queue
within a parent module may be combined into a common queue
of a child through {\bf attach} operations.
 
The {\bf detach} statement causes the specified interaction
point and the one (and only one) to which it has been
attached to be unbound. Any interactions queued at a
connection endpoint at a lower level than the current level
which were sent via the specified interaction point are
appended onto the interaction point's queue before the
interaction points are detached. A variant of this is called
simple detach; in this case no interactions are moved. The
terminate statement uses simple detaches rather than the
conventional detach.

\begin{tabbing}
mm\=mmmmmmmmnn\=mm\=mmmm\=\+\kill
$<${\em output-stmt}$>$\>=\>{\bf output} $<${\em interaction-ref}$>$[$<${\em
ACTUAL-PARAM-LIST}$>$]\\
\mbox{}\\
$<${\em interaction-ref}$>$\>=\>$<${\em ip-ref}$>$ {\bf .} $<${\em
interaction-IDENT}$>$\\
\end{tabbing}

The {\bf output} statement results in the interaction (plus its
parameters, if any) to be appended onto the queue assigned
to the (other) connection endpoint. Outputs made through an
external interaction point may be observed in the target
queue only after the whole issuing transition is completed,
due to the atomicity of transitions.

Interactions are received and removed from queues by
executing transitions with {\bf when} clauses. It may be necessary
to examine a queue without dequeuing an interaction, if the
transition containing the {\bf when} clause also contains a
{\bf provided} clause which refers to one or more of the
interaction paramaters. In this case, the interaction at the
head of the queue (assuming the {\bf when} clause is satisfied)
must first be examined to ensure that the provided clause is
also satisfied, before the interaction can actually be
dequeued.




\newpage
\section[Evaluation and Bug Report]{Evaluation and Bug Report}
As the  {\em PEW} is  still under development, it would be appreciated
if you could complete this form and return it to Graham Wheeler,
DNA Laboratory, at the address on the front of this manual.
All constructive criticism is welcomed.

I liked the following features of the {\em PEW}:

\vspace{8cm}

I disliked the following features of the {\em PEW}:

\vspace{8cm}

I would like to see the following features added:
\newpage
I experienced  the following  bugs (please specify if the bug is
repeatable, what  circumstances caused  it,  what  the  symptoms
where, and how fatal it was):
\end{document}


